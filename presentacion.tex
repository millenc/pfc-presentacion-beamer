\documentclass{beamer}

\mode<presentation>
{
  \usetheme{Madrid}
  \usecolortheme{default}
  \usefonttheme{structurebold}
  \setbeamertemplate{navigation symbols}{}
  \setbeamertemplate{caption}[numbered]
} 

\usepackage[spanish]{babel}
\usepackage[utf8x]{inputenc}

% Gráficos
\usepackage{tikz}
\usetikzlibrary{shapes,shadows,arrows}
\usepackage{pgfplots}
\pgfplotsset{compat=newest}
\usepackage{float}
\usepackage{subcaption}

% Captions
\setbeamertemplate{caption}{\raggedright\insertcaption\par}

\title[Detección de riesgos ambientales]{Aplicación de un método de interpolación basado en índices de solapamiento a la detección de riesgos ambientales}
\author{Mikel Pintor Araus}
\institute[UPNA]
{
  Universidad Pública de Navarra\\
  Escuela Técnica Superior de Ingenieros Industriales y de Telecomunicación
}
\date{23 de julio 2014}

\begin{document}
 
\begin{frame}
  \titlepage
\end{frame}

\begin{frame}{Índice}  
  \tableofcontents
\end{frame}

\section{Teoría de conjuntos difusos}

\begin{frame}{Conjuntos difusos}
      \begin{block}{Definición: Conjunto difuso}
      Dado un conjunto de referencia (o universo) \emph{U}, un \emph{conjunto difuso} \emph{A} sobre \emph{U} es un conjunto tal que:\\
      \begin{equation}
      A=\{(u_{i},\mu_{A}(u_{i}))\arrowvert u_{i} \in U\}
      \end{equation}
      donde \begin{math}\mu_{A}:U\rightarrow[0,1]\end{math} es la \emph{función de pertenencia}\index{función de pertenencia} (o \emph{grado de pertenencia}) de \emph{A}.
      \end{block}
      	\begin{itemize}
      	\item Introducidos por L.A. Zadeh en 1965.
      	\item Extensión de los conjuntos clásicos.
      	\item Permiten modelar información vaga o imprecisa.
      	\end{itemize}
\end{frame}

\begin{frame}{T-normas}
	  Las \emph{t-normas} son una clase de funciones que generalizan el mínimo ($\min(x,y)$) y ,por tanto, la conjunción clásica ($x\wedge y$):
      \begin{block}{Definición: T-norma}
		Una t-norma es una operación binaria \emph{T} en el intervalo $[0,1]$ que es conmutativa, asociativa, monótona y tiene el valor \emph{1} como elemento neutro. Es decir, una función $T : [0,1]^2 \rightarrow [0,1]$ tal que $\forall x,y,z \in [0,1]$:
		\begin{enumerate}
		   \item $T(x,y) = T(y,x)$ (Conmutatividad)
		   \item $T(x,T(y,z)) = T(T(x,y),z)$ (Asociatividad)
		   \item $T(x,y) \leq T(x,z)$ cuando $y \leq z$ (Monotonía)
		   \item $T(x,1) = x$ (Elemento neutro)
		  \end{enumerate}
      \end{block}
\end{frame}

\begin{frame}{T-normas: algunos ejemplos}
	\begin{itemize}
		\item \bfseries Mínimo o t-norma de Gödel: $T_{G}(x,y) = \min\{x,y\}$
		\item \bfseries Producto: $T_{P}(x,y) = x \cdot y$
		\item \bfseries \L{}ukasiewicz: $T_{L}(x,y) = \max\{x+y-1,0\}$
	\end{itemize}
	\vspace{0.2cm}
	\newlength\figureheight
    \newlength\figurewidth 
	\setlength\figureheight{2cm}
	\setlength\figurewidth{3cm}
	\input{figures/t-norms/min-3dplot.tikz}
	\setlength\figureheight{2cm}
	\setlength\figurewidth{3cm}
	% This file was created by matlab2tikz v0.4.7 (commit 4d7ae5c4fd0932fb51051d86111bc23ed23e4580) running on MATLAB 8.0.
% Copyright (c) 2008--2014, Nico Schlömer <nico.schloemer@gmail.com>
% All rights reserved.
% Minimal pgfplots version: 1.3
% 
\begin{tikzpicture}

\begin{axis}[%
width=\figurewidth,
height=\figureheight,
view={-37.5}{30},
scale only axis,
xmin=0,
xmax=1,
xlabel={x},
xmajorgrids,
ymin=0,
ymax=1,
ylabel={y},
ymajorgrids,
zmin=0,
zmax=1,
zlabel={$\text{T}_{\text{P}}\text{(x,y)}$},
zmajorgrids,
axis x line*=bottom,
axis y line*=left,
axis z line*=left
]

\addplot3[%
surf,
shader=faceted,
draw=black,
colormap/jet,
mesh/rows=11]
table[row sep=crcr,header=false] {0	0	0\\
0	0.1	0\\
0	0.2	0\\
0	0.3	0\\
0	0.4	0\\
0	0.5	0\\
0	0.6	0\\
0	0.7	0\\
0	0.8	0\\
0	0.9	0\\
0	1	0\\
0.1	0	0\\
0.1	0.1	0.01\\
0.1	0.2	0.02\\
0.1	0.3	0.03\\
0.1	0.4	0.04\\
0.1	0.5	0.05\\
0.1	0.6	0.06\\
0.1	0.7	0.07\\
0.1	0.8	0.08\\
0.1	0.9	0.09\\
0.1	1	0.1\\
0.2	0	0\\
0.2	0.1	0.02\\
0.2	0.2	0.04\\
0.2	0.3	0.06\\
0.2	0.4	0.08\\
0.2	0.5	0.1\\
0.2	0.6	0.12\\
0.2	0.7	0.14\\
0.2	0.8	0.16\\
0.2	0.9	0.18\\
0.2	1	0.2\\
0.3	0	0\\
0.3	0.1	0.03\\
0.3	0.2	0.06\\
0.3	0.3	0.09\\
0.3	0.4	0.12\\
0.3	0.5	0.15\\
0.3	0.6	0.18\\
0.3	0.7	0.21\\
0.3	0.8	0.24\\
0.3	0.9	0.27\\
0.3	1	0.3\\
0.4	0	0\\
0.4	0.1	0.04\\
0.4	0.2	0.08\\
0.4	0.3	0.12\\
0.4	0.4	0.16\\
0.4	0.5	0.2\\
0.4	0.6	0.24\\
0.4	0.7	0.28\\
0.4	0.8	0.32\\
0.4	0.9	0.36\\
0.4	1	0.4\\
0.5	0	0\\
0.5	0.1	0.05\\
0.5	0.2	0.1\\
0.5	0.3	0.15\\
0.5	0.4	0.2\\
0.5	0.5	0.25\\
0.5	0.6	0.3\\
0.5	0.7	0.35\\
0.5	0.8	0.4\\
0.5	0.9	0.45\\
0.5	1	0.5\\
0.6	0	0\\
0.6	0.1	0.06\\
0.6	0.2	0.12\\
0.6	0.3	0.18\\
0.6	0.4	0.24\\
0.6	0.5	0.3\\
0.6	0.6	0.36\\
0.6	0.7	0.42\\
0.6	0.8	0.48\\
0.6	0.9	0.54\\
0.6	1	0.6\\
0.7	0	0\\
0.7	0.1	0.07\\
0.7	0.2	0.14\\
0.7	0.3	0.21\\
0.7	0.4	0.28\\
0.7	0.5	0.35\\
0.7	0.6	0.42\\
0.7	0.7	0.49\\
0.7	0.8	0.56\\
0.7	0.9	0.63\\
0.7	1	0.7\\
0.8	0	0\\
0.8	0.1	0.08\\
0.8	0.2	0.16\\
0.8	0.3	0.24\\
0.8	0.4	0.32\\
0.8	0.5	0.4\\
0.8	0.6	0.48\\
0.8	0.7	0.56\\
0.8	0.8	0.64\\
0.8	0.9	0.72\\
0.8	1	0.8\\
0.9	0	0\\
0.9	0.1	0.09\\
0.9	0.2	0.18\\
0.9	0.3	0.27\\
0.9	0.4	0.36\\
0.9	0.5	0.45\\
0.9	0.6	0.54\\
0.9	0.7	0.63\\
0.9	0.8	0.72\\
0.9	0.9	0.81\\
0.9	1	0.9\\
1	0	0\\
1	0.1	0.1\\
1	0.2	0.2\\
1	0.3	0.3\\
1	0.4	0.4\\
1	0.5	0.5\\
1	0.6	0.6\\
1	0.7	0.7\\
1	0.8	0.8\\
1	0.9	0.9\\
1	1	1\\
};
\end{axis}
\end{tikzpicture}%
	\centering
	\setlength\figureheight{2cm}
	\setlength\figurewidth{3cm}
	% This file was created by matlab2tikz v0.4.7 (commit 4d7ae5c4fd0932fb51051d86111bc23ed23e4580) running on MATLAB 8.0.
% Copyright (c) 2008--2014, Nico Schlömer <nico.schloemer@gmail.com>
% All rights reserved.
% Minimal pgfplots version: 1.3
% 
\begin{tikzpicture}

\begin{axis}[%
width=\figurewidth,
height=\figureheight,
view={-37.5}{30},
scale only axis,
xmin=0,
xmax=1,
xlabel={x},
xmajorgrids,
ymin=0,
ymax=1,
ylabel={y},
ymajorgrids,
zmin=0,
zmax=1,
zlabel={$\text{T}_{\text{L}}\text{(x,y)}$},
zmajorgrids,
axis x line*=bottom,
axis y line*=left,
axis z line*=left
]

\addplot3[%
surf,
shader=faceted,
draw=black,
colormap/jet,
mesh/rows=11]
table[row sep=crcr,header=false] {0	0	0\\
0	0.1	0\\
0	0.2	0\\
0	0.3	0\\
0	0.4	0\\
0	0.5	0\\
0	0.6	0\\
0	0.7	0\\
0	0.8	0\\
0	0.9	0\\
0	1	0\\
0.1	0	0\\
0.1	0.1	0\\
0.1	0.2	0\\
0.1	0.3	0\\
0.1	0.4	0\\
0.1	0.5	0\\
0.1	0.6	0\\
0.1	0.7	0\\
0.1	0.8	0\\
0.1	0.9	0\\
0.1	1	0.1\\
0.2	0	0\\
0.2	0.1	0\\
0.2	0.2	0\\
0.2	0.3	0\\
0.2	0.4	0\\
0.2	0.5	0\\
0.2	0.6	0\\
0.2	0.7	0\\
0.2	0.8	0\\
0.2	0.9	0.1\\
0.2	1	0.2\\
0.3	0	0\\
0.3	0.1	0\\
0.3	0.2	0\\
0.3	0.3	0\\
0.3	0.4	0\\
0.3	0.5	0\\
0.3	0.6	0\\
0.3	0.7	0\\
0.3	0.8	0.1\\
0.3	0.9	0.2\\
0.3	1	0.3\\
0.4	0	0\\
0.4	0.1	0\\
0.4	0.2	0\\
0.4	0.3	0\\
0.4	0.4	0\\
0.4	0.5	0\\
0.4	0.6	0\\
0.4	0.7	0.1\\
0.4	0.8	0.2\\
0.4	0.9	0.3\\
0.4	1	0.4\\
0.5	0	0\\
0.5	0.1	0\\
0.5	0.2	0\\
0.5	0.3	0\\
0.5	0.4	0\\
0.5	0.5	0\\
0.5	0.6	0.1\\
0.5	0.7	0.2\\
0.5	0.8	0.3\\
0.5	0.9	0.4\\
0.5	1	0.5\\
0.6	0	0\\
0.6	0.1	0\\
0.6	0.2	0\\
0.6	0.3	0\\
0.6	0.4	0\\
0.6	0.5	0.1\\
0.6	0.6	0.2\\
0.6	0.7	0.3\\
0.6	0.8	0.4\\
0.6	0.9	0.5\\
0.6	1	0.6\\
0.7	0	0\\
0.7	0.1	0\\
0.7	0.2	0\\
0.7	0.3	0\\
0.7	0.4	0.1\\
0.7	0.5	0.2\\
0.7	0.6	0.3\\
0.7	0.7	0.4\\
0.7	0.8	0.5\\
0.7	0.9	0.6\\
0.7	1	0.7\\
0.8	0	0\\
0.8	0.1	0\\
0.8	0.2	0\\
0.8	0.3	0.1\\
0.8	0.4	0.2\\
0.8	0.5	0.3\\
0.8	0.6	0.4\\
0.8	0.7	0.5\\
0.8	0.8	0.6\\
0.8	0.9	0.7\\
0.8	1	0.8\\
0.9	0	0\\
0.9	0.1	0\\
0.9	0.2	0.1\\
0.9	0.3	0.2\\
0.9	0.4	0.3\\
0.9	0.5	0.4\\
0.9	0.6	0.5\\
0.9	0.7	0.6\\
0.9	0.8	0.7\\
0.9	0.9	0.8\\
0.9	1	0.9\\
1	0	0\\
1	0.1	0.1\\
1	0.2	0.2\\
1	0.3	0.3\\
1	0.4	0.4\\
1	0.5	0.5\\
1	0.6	0.6\\
1	0.7	0.7\\
1	0.8	0.8\\
1	0.9	0.9\\
1	1	1\\
};
\end{axis}
\end{tikzpicture}%
\end{frame}

\begin{frame}{Operadores de agregación}
	\begin{block}{Definición: operador de agregación}
	Una función $M : [a,b]^{n} \rightarrow [a,b]$ es un operador de agregación si es monótona y no decreciente en cada una de sus componentes y además cumple que $M(a, a, \cdots,a) = a$ y $M(b, b, \cdots,b) = b$.
	\end{block}
	Algunos ejemplos de operadores de agregación:
	\begin{itemize}
		\item \bfseries Media aritmética: $M(x_{1},x_{2},\cdots,x_{n}) = \frac{1}{n}\sum\limits_{i=1}^{n}x_{i}$
		\item \bfseries Media geométrica: $M(x_{1},x_{2},\cdots,x_{n}) = (\prod\limits_{i=1}^{n}x_{i})^{\frac{1}{n}}$
		\item \bfseries Mediana: \normalfont Se toma el elemento central del conjunto ordenado de argumentos.
		\item \bfseries Máximo: $M(x_{1},x_{2},\cdots,x_{n}) = \max(x_{1},x_{2},\cdots,x_{n})$
		\item \bfseries Mínimo: $M(x_{1},x_{2},\cdots,x_{n}) = \min(x_{1},x_{2},\cdots,x_{n})$
	\end{itemize}
\end{frame}

\begin{frame}{Funciones de solapamiento}
Las funciones de solapamiento generalizan los operadores de intersección tales como el mínimo o, en general, las t-normas:
	\begin{block}{Definición: función de solapamiento}
	Una función de solapamiento es una función $G_{O} : [0,1]^{2} \rightarrow [0,1]$ que cumple:
	\begin{enumerate}
	   \item $G_{O}(x,y) = G_{O}(y,x) \;\; \forall \; x,y \in [0,1]$ 
	   \item $G_{O}(x,y) = 0$ si y sólo si $x \cdot y = 0$
	   \item $G_{O}(x,y) = 1$ si y sólo si $x \cdot y = 1$
	   \item $G_{O}$ es creciente
	   \item $G_{O}$ es continua
	\end{enumerate}
	\end{block}
\end{frame}

\begin{frame}{Funciones de solapamiento (cont.)}
	Algunos ejemplos de funciones de solapamiento:
	\begin{figure}
	\begin{subfigure}{0.45\textwidth}
		\caption{$G_{O}(x,y) = \min\{x,y\}$}
		\setlength\figureheight{3cm}
		\setlength\figurewidth{4cm}
		\input{figures/overlap-functions/overlap-min-3dplot.tikz}
	\end{subfigure}
	\qquad
	\begin{subfigure}{0.45\textwidth}
		\caption{$G_{O}(x,y) = \sqrt{x \cdot y}$}
		\setlength\figureheight{3cm}
		\setlength\figurewidth{4cm}
		% This file was created by matlab2tikz v0.4.7 (commit 921463bb921f990ffc6a7f597361910de95829d7) running on MATLAB 8.0.
% Copyright (c) 2008--2014, Nico Schlömer <nico.schloemer@gmail.com>
% All rights reserved.
% Minimal pgfplots version: 1.3
% 
\begin{tikzpicture}

\begin{axis}[%
width=\figurewidth,
height=\figureheight,
view={-37.5}{30},
scale only axis,
xmin=0,
xmax=1,
xlabel={x},
xmajorgrids,
ymin=0,
ymax=1,
ylabel={y},
ymajorgrids,
zmin=0,
zmax=1,
zlabel={$\text{G}_{\text{O}}\text{(x,y)}$},
zmajorgrids,
axis x line*=bottom,
axis y line*=left,
axis z line*=left
]

\addplot3[%
surf,
shader=faceted,
draw=black,
colormap/jet,
mesh/rows=11]
table[row sep=crcr,header=false] {0	0	0\\
0	0.1	0\\
0	0.2	0\\
0	0.3	0\\
0	0.4	0\\
0	0.5	0\\
0	0.6	0\\
0	0.7	0\\
0	0.8	0\\
0	0.9	0\\
0	1	0\\
0.1	0	0\\
0.1	0.1	0.1\\
0.1	0.2	0.14142135623731\\
0.1	0.3	0.173205080756888\\
0.1	0.4	0.2\\
0.1	0.5	0.223606797749979\\
0.1	0.6	0.244948974278318\\
0.1	0.7	0.264575131106459\\
0.1	0.8	0.282842712474619\\
0.1	0.9	0.3\\
0.1	1	0.316227766016838\\
0.2	0	0\\
0.2	0.1	0.14142135623731\\
0.2	0.2	0.2\\
0.2	0.3	0.244948974278318\\
0.2	0.4	0.282842712474619\\
0.2	0.5	0.316227766016838\\
0.2	0.6	0.346410161513775\\
0.2	0.7	0.374165738677394\\
0.2	0.8	0.4\\
0.2	0.9	0.424264068711929\\
0.2	1	0.447213595499958\\
0.3	0	0\\
0.3	0.1	0.173205080756888\\
0.3	0.2	0.244948974278318\\
0.3	0.3	0.3\\
0.3	0.4	0.346410161513776\\
0.3	0.5	0.387298334620742\\
0.3	0.6	0.424264068711929\\
0.3	0.7	0.458257569495584\\
0.3	0.8	0.489897948556636\\
0.3	0.9	0.519615242270663\\
0.3	1	0.547722557505166\\
0.4	0	0\\
0.4	0.1	0.2\\
0.4	0.2	0.282842712474619\\
0.4	0.3	0.346410161513776\\
0.4	0.4	0.4\\
0.4	0.5	0.447213595499958\\
0.4	0.6	0.489897948556636\\
0.4	0.7	0.529150262212918\\
0.4	0.8	0.565685424949238\\
0.4	0.9	0.6\\
0.4	1	0.632455532033676\\
0.5	0	0\\
0.5	0.1	0.223606797749979\\
0.5	0.2	0.316227766016838\\
0.5	0.3	0.387298334620742\\
0.5	0.4	0.447213595499958\\
0.5	0.5	0.5\\
0.5	0.6	0.547722557505166\\
0.5	0.7	0.591607978309962\\
0.5	0.8	0.632455532033676\\
0.5	0.9	0.670820393249937\\
0.5	1	0.707106781186548\\
0.6	0	0\\
0.6	0.1	0.244948974278318\\
0.6	0.2	0.346410161513775\\
0.6	0.3	0.424264068711929\\
0.6	0.4	0.489897948556636\\
0.6	0.5	0.547722557505166\\
0.6	0.6	0.6\\
0.6	0.7	0.648074069840786\\
0.6	0.8	0.692820323027551\\
0.6	0.9	0.734846922834953\\
0.6	1	0.774596669241483\\
0.7	0	0\\
0.7	0.1	0.264575131106459\\
0.7	0.2	0.374165738677394\\
0.7	0.3	0.458257569495584\\
0.7	0.4	0.529150262212918\\
0.7	0.5	0.591607978309962\\
0.7	0.6	0.648074069840786\\
0.7	0.7	0.7\\
0.7	0.8	0.748331477354788\\
0.7	0.9	0.793725393319377\\
0.7	1	0.836660026534076\\
0.8	0	0\\
0.8	0.1	0.282842712474619\\
0.8	0.2	0.4\\
0.8	0.3	0.489897948556636\\
0.8	0.4	0.565685424949238\\
0.8	0.5	0.632455532033676\\
0.8	0.6	0.692820323027551\\
0.8	0.7	0.748331477354788\\
0.8	0.8	0.8\\
0.8	0.9	0.848528137423857\\
0.8	1	0.894427190999916\\
0.9	0	0\\
0.9	0.1	0.3\\
0.9	0.2	0.424264068711929\\
0.9	0.3	0.519615242270663\\
0.9	0.4	0.6\\
0.9	0.5	0.670820393249937\\
0.9	0.6	0.734846922834953\\
0.9	0.7	0.793725393319377\\
0.9	0.8	0.848528137423857\\
0.9	0.9	0.9\\
0.9	1	0.948683298050514\\
1	0	0\\
1	0.1	0.316227766016838\\
1	0.2	0.447213595499958\\
1	0.3	0.547722557505166\\
1	0.4	0.632455532033676\\
1	0.5	0.707106781186548\\
1	0.6	0.774596669241483\\
1	0.7	0.836660026534076\\
1	0.8	0.894427190999916\\
1	0.9	0.948683298050514\\
1	1	1\\
};
\end{axis}
\end{tikzpicture}%
	\end{subfigure}
	\end{figure}
\end{frame}

\begin{frame}{Funciones de solapamiento (cont.)}
	\begin{block}{Teorema: Construcción de funciones de solapamiento}
		Una función $G_{O} : [0,1]^{2} \rightarrow [0,1]$ es una función de solapamiento si y sólo si puede ser expresada como:
		\begin{equation}
		G_{O}(x,y) = \frac{f(x,y)}{f(x,y) + h(x,y)}
		\end{equation}
		para cualesquiera $f,h: [0,1]^{2} \rightarrow [0,1]$ tales que:
		\begin{enumerate}
		   \item f y h son simétricas;
		   \item f es no decreciente y h es no creciente;
		   \item $f(x,y) = 0$ si y sólo si $min(x.y) = 0$;
		   \item $h(x,y) = 0$ si y sólo si $min(x.y) = 1$;
		   \item f y h son continuas
		\end{enumerate}
	\end{block}
\end{frame}

\begin{frame}{Funciones de solapamiento (cont.)}
	Algunos ejemplos de construcción de funciones de solapamiento: 
	\vspace{0.5cm}
	\begin{enumerate}
		\item Si $f(x,y)=min(x,y)$ y $h(x,y) = max(1-x,1-y)$ entonces: 
		\begin{equation}
		G_{O}(x,y) = min(x,y)
		\end{equation}
		\item Si $f(x,y)=\sqrt{x \cdot y}$ y $h(x,y) = max(1-x,1-y)$ entonces: 
		\begin{equation}G_{O}(x,y) = \frac{\sqrt{x \cdot y}}{\sqrt{x \cdot y} + max(1-x,1-y)}
		\end{equation}
		\item Si $f(x,y)=\sqrt{x \cdot y}$ y $h(x,y) = 1 - x \cdot y $ entonces: 
		\begin{equation}G_{O}(x,y) = \frac{\sqrt{x \cdot y}}{\sqrt{x \cdot y} + 1 - x \cdot y}
		\end{equation}
	\end{enumerate}
\end{frame}

\end{document}