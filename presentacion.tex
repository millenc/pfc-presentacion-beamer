\documentclass{beamer}

\mode<presentation>
{
  \usetheme{Madrid}
  \usecolortheme{default}
  \usefonttheme{structurebold}
  \setbeamertemplate{navigation symbols}{}
  \setbeamertemplate{caption}[numbered]
} 

\usepackage[spanish]{babel}
\usepackage[utf8x]{inputenc}

% Gráficos
\usepackage{tikz}
\usetikzlibrary{shapes,shadows,arrows}
\usepackage{pgfplots}
\pgfplotsset{compat=newest}
\usepackage{float}
\setlength{\abovecaptionskip}{0pt plus 0pt minus 0pt}
\setlength{\belowcaptionskip}{0pt plus 0pt minus 0pt}
\usepackage{subcaption}

% Algoritmos
\usepackage[linesnumbered]{algorithm2e}
\renewcommand*{\algorithmcfname}{Algoritmo}
\let\oldnl\nl% Store \nl in \oldnl
\newcommand{\nonl}{\renewcommand{\nl}{\let\nl\oldnl}}

% Tablas
\usepackage{multirow}
\usepackage{hhline}

% Captions
\setbeamertemplate{caption}{\raggedright\insertcaption\par}

\title[Detección de riesgos ambientales]{Aplicación de un método de interpolación basado en índices de solapamiento a la detección de riesgos ambientales}
\author{Mikel Pintor Araus}
\institute[UPNA]
{
  Universidad Pública de Navarra\\
  Escuela Técnica Superior de Ingenieros Industriales y de Telecomunicación
}
\date{23 de julio 2014}

\begin{document}
 
\begin{frame}
  \titlepage
\end{frame}

\begin{frame}{Índice}  
  \tableofcontents
\end{frame}

\section{Teoría de conjuntos difusos}

\subsection{Conjuntos difusos}
\begin{frame}{Conjuntos difusos}
      \begin{block}{Definición: Conjunto difuso}
      Dado un conjunto de referencia (o universo) \emph{U}, un \emph{conjunto difuso} \emph{A} sobre \emph{U} es un conjunto tal que:\\
      \begin{equation}
      A=\{(u_{i},\mu_{A}(u_{i}))\arrowvert u_{i} \in U\}
      \end{equation}
      donde \begin{math}\mu_{A}:U\rightarrow[0,1]\end{math} es la \emph{función de pertenencia}\index{función de pertenencia} (o \emph{grado de pertenencia}) de \emph{A}.
      \end{block}
      	\begin{itemize}
      	\item Introducidos por L.A. Zadeh en 1965.
      	\item Extensión de los conjuntos clásicos.
      	\item Permiten modelar información vaga o imprecisa.
      	\end{itemize}
\end{frame}

\subsection{T-normas y operadores de agregación}
\begin{frame}{T-normas}
	  Las \emph{t-normas} son una clase de funciones que generalizan el mínimo ($\min(x,y)$) y ,por tanto, la conjunción clásica ($x\wedge y$):
      \begin{block}{Definición: T-norma}
		Una t-norma es una operación binaria \emph{T} en el intervalo $[0,1]$ que es conmutativa, asociativa, monótona y tiene el valor \emph{1} como elemento neutro. Es decir, una función $T : [0,1]^2 \rightarrow [0,1]$ tal que $\forall x,y,z \in [0,1]$:
		\begin{enumerate}
		   \item $T(x,y) = T(y,x)$ (Conmutatividad)
		   \item $T(x,T(y,z)) = T(T(x,y),z)$ (Asociatividad)
		   \item $T(x,y) \leq T(x,z)$ cuando $y \leq z$ (Monotonía)
		   \item $T(x,1) = x$ (Elemento neutro)
		  \end{enumerate}
      \end{block}
\end{frame}

\begin{frame}{T-normas: algunos ejemplos}
	\begin{itemize}
		\item \bfseries Mínimo o t-norma de Gödel: $T_{G}(x,y) = \min\{x,y\}$
		\item \bfseries Producto: $T_{P}(x,y) = x \cdot y$
		\item \bfseries \L{}ukasiewicz: $T_{L}(x,y) = \max\{x+y-1,0\}$
	\end{itemize}
	\vspace{0.2cm}
	\newlength\figureheight
    \newlength\figurewidth 
	\setlength\figureheight{2cm}
	\setlength\figurewidth{3cm}
	% This file was created by matlab2tikz v0.4.7 (commit 4d7ae5c4fd0932fb51051d86111bc23ed23e4580) running on MATLAB 8.0.
% Copyright (c) 2008--2014, Nico Schlömer <nico.schloemer@gmail.com>
% All rights reserved.
% Minimal pgfplots version: 1.3
% 
\begin{tikzpicture}

\begin{axis}[%
width=\figurewidth,
height=\figureheight,
view={-37.5}{30},
scale only axis,
xmin=0,
xmax=1,
xlabel={x},
xmajorgrids,
ymin=0,
ymax=1,
ylabel={y},
ymajorgrids,
zmin=0,
zmax=1,
zlabel={$\text{T}_{\text{G}}\text{(x,y)}$},
zmajorgrids,
axis x line*=bottom,
axis y line*=left,
axis z line*=left
]

\addplot3[%
surf,
shader=faceted,
draw=black,
colormap/jet,
mesh/rows=11]
table[row sep=crcr,header=false] {0	0	0\\
0	0.1	0\\
0	0.2	0\\
0	0.3	0\\
0	0.4	0\\
0	0.5	0\\
0	0.6	0\\
0	0.7	0\\
0	0.8	0\\
0	0.9	0\\
0	1	0\\
0.1	0	0\\
0.1	0.1	0.1\\
0.1	0.2	0.1\\
0.1	0.3	0.1\\
0.1	0.4	0.1\\
0.1	0.5	0.1\\
0.1	0.6	0.1\\
0.1	0.7	0.1\\
0.1	0.8	0.1\\
0.1	0.9	0.1\\
0.1	1	0.1\\
0.2	0	0\\
0.2	0.1	0.1\\
0.2	0.2	0.2\\
0.2	0.3	0.2\\
0.2	0.4	0.2\\
0.2	0.5	0.2\\
0.2	0.6	0.2\\
0.2	0.7	0.2\\
0.2	0.8	0.2\\
0.2	0.9	0.2\\
0.2	1	0.2\\
0.3	0	0\\
0.3	0.1	0.1\\
0.3	0.2	0.2\\
0.3	0.3	0.3\\
0.3	0.4	0.3\\
0.3	0.5	0.3\\
0.3	0.6	0.3\\
0.3	0.7	0.3\\
0.3	0.8	0.3\\
0.3	0.9	0.3\\
0.3	1	0.3\\
0.4	0	0\\
0.4	0.1	0.1\\
0.4	0.2	0.2\\
0.4	0.3	0.3\\
0.4	0.4	0.4\\
0.4	0.5	0.4\\
0.4	0.6	0.4\\
0.4	0.7	0.4\\
0.4	0.8	0.4\\
0.4	0.9	0.4\\
0.4	1	0.4\\
0.5	0	0\\
0.5	0.1	0.1\\
0.5	0.2	0.2\\
0.5	0.3	0.3\\
0.5	0.4	0.4\\
0.5	0.5	0.5\\
0.5	0.6	0.5\\
0.5	0.7	0.5\\
0.5	0.8	0.5\\
0.5	0.9	0.5\\
0.5	1	0.5\\
0.6	0	0\\
0.6	0.1	0.1\\
0.6	0.2	0.2\\
0.6	0.3	0.3\\
0.6	0.4	0.4\\
0.6	0.5	0.5\\
0.6	0.6	0.6\\
0.6	0.7	0.6\\
0.6	0.8	0.6\\
0.6	0.9	0.6\\
0.6	1	0.6\\
0.7	0	0\\
0.7	0.1	0.1\\
0.7	0.2	0.2\\
0.7	0.3	0.3\\
0.7	0.4	0.4\\
0.7	0.5	0.5\\
0.7	0.6	0.6\\
0.7	0.7	0.7\\
0.7	0.8	0.7\\
0.7	0.9	0.7\\
0.7	1	0.7\\
0.8	0	0\\
0.8	0.1	0.1\\
0.8	0.2	0.2\\
0.8	0.3	0.3\\
0.8	0.4	0.4\\
0.8	0.5	0.5\\
0.8	0.6	0.6\\
0.8	0.7	0.7\\
0.8	0.8	0.8\\
0.8	0.9	0.8\\
0.8	1	0.8\\
0.9	0	0\\
0.9	0.1	0.1\\
0.9	0.2	0.2\\
0.9	0.3	0.3\\
0.9	0.4	0.4\\
0.9	0.5	0.5\\
0.9	0.6	0.6\\
0.9	0.7	0.7\\
0.9	0.8	0.8\\
0.9	0.9	0.9\\
0.9	1	0.9\\
1	0	0\\
1	0.1	0.1\\
1	0.2	0.2\\
1	0.3	0.3\\
1	0.4	0.4\\
1	0.5	0.5\\
1	0.6	0.6\\
1	0.7	0.7\\
1	0.8	0.8\\
1	0.9	0.9\\
1	1	1\\
};
\end{axis}
\end{tikzpicture}%
	\setlength\figureheight{2cm}
	\setlength\figurewidth{3cm}
	% This file was created by matlab2tikz v0.4.7 (commit 4d7ae5c4fd0932fb51051d86111bc23ed23e4580) running on MATLAB 8.0.
% Copyright (c) 2008--2014, Nico Schlömer <nico.schloemer@gmail.com>
% All rights reserved.
% Minimal pgfplots version: 1.3
% 
\begin{tikzpicture}

\begin{axis}[%
width=\figurewidth,
height=\figureheight,
view={-37.5}{30},
scale only axis,
xmin=0,
xmax=1,
xlabel={x},
xmajorgrids,
ymin=0,
ymax=1,
ylabel={y},
ymajorgrids,
zmin=0,
zmax=1,
zlabel={$\text{T}_{\text{P}}\text{(x,y)}$},
zmajorgrids,
axis x line*=bottom,
axis y line*=left,
axis z line*=left
]

\addplot3[%
surf,
shader=faceted,
draw=black,
colormap/jet,
mesh/rows=11]
table[row sep=crcr,header=false] {0	0	0\\
0	0.1	0\\
0	0.2	0\\
0	0.3	0\\
0	0.4	0\\
0	0.5	0\\
0	0.6	0\\
0	0.7	0\\
0	0.8	0\\
0	0.9	0\\
0	1	0\\
0.1	0	0\\
0.1	0.1	0.01\\
0.1	0.2	0.02\\
0.1	0.3	0.03\\
0.1	0.4	0.04\\
0.1	0.5	0.05\\
0.1	0.6	0.06\\
0.1	0.7	0.07\\
0.1	0.8	0.08\\
0.1	0.9	0.09\\
0.1	1	0.1\\
0.2	0	0\\
0.2	0.1	0.02\\
0.2	0.2	0.04\\
0.2	0.3	0.06\\
0.2	0.4	0.08\\
0.2	0.5	0.1\\
0.2	0.6	0.12\\
0.2	0.7	0.14\\
0.2	0.8	0.16\\
0.2	0.9	0.18\\
0.2	1	0.2\\
0.3	0	0\\
0.3	0.1	0.03\\
0.3	0.2	0.06\\
0.3	0.3	0.09\\
0.3	0.4	0.12\\
0.3	0.5	0.15\\
0.3	0.6	0.18\\
0.3	0.7	0.21\\
0.3	0.8	0.24\\
0.3	0.9	0.27\\
0.3	1	0.3\\
0.4	0	0\\
0.4	0.1	0.04\\
0.4	0.2	0.08\\
0.4	0.3	0.12\\
0.4	0.4	0.16\\
0.4	0.5	0.2\\
0.4	0.6	0.24\\
0.4	0.7	0.28\\
0.4	0.8	0.32\\
0.4	0.9	0.36\\
0.4	1	0.4\\
0.5	0	0\\
0.5	0.1	0.05\\
0.5	0.2	0.1\\
0.5	0.3	0.15\\
0.5	0.4	0.2\\
0.5	0.5	0.25\\
0.5	0.6	0.3\\
0.5	0.7	0.35\\
0.5	0.8	0.4\\
0.5	0.9	0.45\\
0.5	1	0.5\\
0.6	0	0\\
0.6	0.1	0.06\\
0.6	0.2	0.12\\
0.6	0.3	0.18\\
0.6	0.4	0.24\\
0.6	0.5	0.3\\
0.6	0.6	0.36\\
0.6	0.7	0.42\\
0.6	0.8	0.48\\
0.6	0.9	0.54\\
0.6	1	0.6\\
0.7	0	0\\
0.7	0.1	0.07\\
0.7	0.2	0.14\\
0.7	0.3	0.21\\
0.7	0.4	0.28\\
0.7	0.5	0.35\\
0.7	0.6	0.42\\
0.7	0.7	0.49\\
0.7	0.8	0.56\\
0.7	0.9	0.63\\
0.7	1	0.7\\
0.8	0	0\\
0.8	0.1	0.08\\
0.8	0.2	0.16\\
0.8	0.3	0.24\\
0.8	0.4	0.32\\
0.8	0.5	0.4\\
0.8	0.6	0.48\\
0.8	0.7	0.56\\
0.8	0.8	0.64\\
0.8	0.9	0.72\\
0.8	1	0.8\\
0.9	0	0\\
0.9	0.1	0.09\\
0.9	0.2	0.18\\
0.9	0.3	0.27\\
0.9	0.4	0.36\\
0.9	0.5	0.45\\
0.9	0.6	0.54\\
0.9	0.7	0.63\\
0.9	0.8	0.72\\
0.9	0.9	0.81\\
0.9	1	0.9\\
1	0	0\\
1	0.1	0.1\\
1	0.2	0.2\\
1	0.3	0.3\\
1	0.4	0.4\\
1	0.5	0.5\\
1	0.6	0.6\\
1	0.7	0.7\\
1	0.8	0.8\\
1	0.9	0.9\\
1	1	1\\
};
\end{axis}
\end{tikzpicture}%
	\centering
	\setlength\figureheight{2cm}
	\setlength\figurewidth{3cm}
	% This file was created by matlab2tikz v0.4.7 (commit 4d7ae5c4fd0932fb51051d86111bc23ed23e4580) running on MATLAB 8.0.
% Copyright (c) 2008--2014, Nico Schlömer <nico.schloemer@gmail.com>
% All rights reserved.
% Minimal pgfplots version: 1.3
% 
\begin{tikzpicture}

\begin{axis}[%
width=\figurewidth,
height=\figureheight,
view={-37.5}{30},
scale only axis,
xmin=0,
xmax=1,
xlabel={x},
xmajorgrids,
ymin=0,
ymax=1,
ylabel={y},
ymajorgrids,
zmin=0,
zmax=1,
zlabel={$\text{T}_{\text{L}}\text{(x,y)}$},
zmajorgrids,
axis x line*=bottom,
axis y line*=left,
axis z line*=left
]

\addplot3[%
surf,
shader=faceted,
draw=black,
colormap/jet,
mesh/rows=11]
table[row sep=crcr,header=false] {0	0	0\\
0	0.1	0\\
0	0.2	0\\
0	0.3	0\\
0	0.4	0\\
0	0.5	0\\
0	0.6	0\\
0	0.7	0\\
0	0.8	0\\
0	0.9	0\\
0	1	0\\
0.1	0	0\\
0.1	0.1	0\\
0.1	0.2	0\\
0.1	0.3	0\\
0.1	0.4	0\\
0.1	0.5	0\\
0.1	0.6	0\\
0.1	0.7	0\\
0.1	0.8	0\\
0.1	0.9	0\\
0.1	1	0.1\\
0.2	0	0\\
0.2	0.1	0\\
0.2	0.2	0\\
0.2	0.3	0\\
0.2	0.4	0\\
0.2	0.5	0\\
0.2	0.6	0\\
0.2	0.7	0\\
0.2	0.8	0\\
0.2	0.9	0.1\\
0.2	1	0.2\\
0.3	0	0\\
0.3	0.1	0\\
0.3	0.2	0\\
0.3	0.3	0\\
0.3	0.4	0\\
0.3	0.5	0\\
0.3	0.6	0\\
0.3	0.7	0\\
0.3	0.8	0.1\\
0.3	0.9	0.2\\
0.3	1	0.3\\
0.4	0	0\\
0.4	0.1	0\\
0.4	0.2	0\\
0.4	0.3	0\\
0.4	0.4	0\\
0.4	0.5	0\\
0.4	0.6	0\\
0.4	0.7	0.1\\
0.4	0.8	0.2\\
0.4	0.9	0.3\\
0.4	1	0.4\\
0.5	0	0\\
0.5	0.1	0\\
0.5	0.2	0\\
0.5	0.3	0\\
0.5	0.4	0\\
0.5	0.5	0\\
0.5	0.6	0.1\\
0.5	0.7	0.2\\
0.5	0.8	0.3\\
0.5	0.9	0.4\\
0.5	1	0.5\\
0.6	0	0\\
0.6	0.1	0\\
0.6	0.2	0\\
0.6	0.3	0\\
0.6	0.4	0\\
0.6	0.5	0.1\\
0.6	0.6	0.2\\
0.6	0.7	0.3\\
0.6	0.8	0.4\\
0.6	0.9	0.5\\
0.6	1	0.6\\
0.7	0	0\\
0.7	0.1	0\\
0.7	0.2	0\\
0.7	0.3	0\\
0.7	0.4	0.1\\
0.7	0.5	0.2\\
0.7	0.6	0.3\\
0.7	0.7	0.4\\
0.7	0.8	0.5\\
0.7	0.9	0.6\\
0.7	1	0.7\\
0.8	0	0\\
0.8	0.1	0\\
0.8	0.2	0\\
0.8	0.3	0.1\\
0.8	0.4	0.2\\
0.8	0.5	0.3\\
0.8	0.6	0.4\\
0.8	0.7	0.5\\
0.8	0.8	0.6\\
0.8	0.9	0.7\\
0.8	1	0.8\\
0.9	0	0\\
0.9	0.1	0\\
0.9	0.2	0.1\\
0.9	0.3	0.2\\
0.9	0.4	0.3\\
0.9	0.5	0.4\\
0.9	0.6	0.5\\
0.9	0.7	0.6\\
0.9	0.8	0.7\\
0.9	0.9	0.8\\
0.9	1	0.9\\
1	0	0\\
1	0.1	0.1\\
1	0.2	0.2\\
1	0.3	0.3\\
1	0.4	0.4\\
1	0.5	0.5\\
1	0.6	0.6\\
1	0.7	0.7\\
1	0.8	0.8\\
1	0.9	0.9\\
1	1	1\\
};
\end{axis}
\end{tikzpicture}%
\end{frame}

\begin{frame}{Operadores de agregación}
	\begin{block}{Definición: operador de agregación}
	Una función $M : [a,b]^{n} \rightarrow [a,b]$ es un operador de agregación si es monótona y no decreciente en cada una de sus componentes y además cumple que $M(a, a, \cdots,a) = a$ y $M(b, b, \cdots,b) = b$.
	\end{block}
	Algunos ejemplos de operadores de agregación:
	\begin{itemize}
		\item \bfseries Media aritmética: $M(x_{1},x_{2},\cdots,x_{n}) = \frac{1}{n}\sum\limits_{i=1}^{n}x_{i}$
		\item \bfseries Media geométrica: $M(x_{1},x_{2},\cdots,x_{n}) = (\prod\limits_{i=1}^{n}x_{i})^{\frac{1}{n}}$
		\item \bfseries Mediana: \normalfont Se toma el elemento central del conjunto ordenado de argumentos.
		\item \bfseries Máximo: $M(x_{1},x_{2},\cdots,x_{n}) = \max(x_{1},x_{2},\cdots,x_{n})$
		\item \bfseries Mínimo: $M(x_{1},x_{2},\cdots,x_{n}) = \min(x_{1},x_{2},\cdots,x_{n})$
	\end{itemize}
\end{frame}

\subsection{Funciones de solapamiento}
\begin{frame}{Funciones de solapamiento}
Las funciones de solapamiento generalizan los operadores de intersección tales como el mínimo o, en general, las t-normas:
	\begin{block}{Definición: función de solapamiento}
	Una función de solapamiento es una función $G_{O} : [0,1]^{2} \rightarrow [0,1]$ que cumple:
	\begin{enumerate}
	   \item $G_{O}(x,y) = G_{O}(y,x) \;\; \forall \; x,y \in [0,1]$ 
	   \item $G_{O}(x,y) = 0$ si y sólo si $x \cdot y = 0$
	   \item $G_{O}(x,y) = 1$ si y sólo si $x \cdot y = 1$
	   \item $G_{O}$ es creciente
	   \item $G_{O}$ es continua
	\end{enumerate}
	\end{block}
\end{frame}

\begin{frame}{Funciones de solapamiento (cont.)}
	Algunos ejemplos de funciones de solapamiento:
	\begin{figure}
	\begin{subfigure}{0.45\textwidth}
		\caption{$G_{O}(x,y) = \min\{x,y\}$}
		\setlength\figureheight{3cm}
		\setlength\figurewidth{4cm}
		% This file was created by matlab2tikz v0.4.7 (commit 921463bb921f990ffc6a7f597361910de95829d7) running on MATLAB 8.0.
% Copyright (c) 2008--2014, Nico Schlömer <nico.schloemer@gmail.com>
% All rights reserved.
% Minimal pgfplots version: 1.3
% 
\begin{tikzpicture}

\begin{axis}[%
width=\figurewidth,
height=\figureheight,
view={-37.5}{30},
scale only axis,
xmin=0,
xmax=1,
xlabel={x},
xmajorgrids,
ymin=0,
ymax=1,
ylabel={y},
ymajorgrids,
zmin=0,
zmax=1,
zlabel={$\text{G}_{\text{O}}\text{(x,y)}$},
zmajorgrids,
axis x line*=bottom,
axis y line*=left,
axis z line*=left
]

\addplot3[%
surf,
shader=faceted,
draw=black,
colormap/jet,
mesh/rows=11]
table[row sep=crcr,header=false] {0	0	0\\
0	0.1	0\\
0	0.2	0\\
0	0.3	0\\
0	0.4	0\\
0	0.5	0\\
0	0.6	0\\
0	0.7	0\\
0	0.8	0\\
0	0.9	0\\
0	1	0\\
0.1	0	0\\
0.1	0.1	0.1\\
0.1	0.2	0.1\\
0.1	0.3	0.1\\
0.1	0.4	0.1\\
0.1	0.5	0.1\\
0.1	0.6	0.1\\
0.1	0.7	0.1\\
0.1	0.8	0.1\\
0.1	0.9	0.1\\
0.1	1	0.1\\
0.2	0	0\\
0.2	0.1	0.1\\
0.2	0.2	0.2\\
0.2	0.3	0.2\\
0.2	0.4	0.2\\
0.2	0.5	0.2\\
0.2	0.6	0.2\\
0.2	0.7	0.2\\
0.2	0.8	0.2\\
0.2	0.9	0.2\\
0.2	1	0.2\\
0.3	0	0\\
0.3	0.1	0.1\\
0.3	0.2	0.2\\
0.3	0.3	0.3\\
0.3	0.4	0.3\\
0.3	0.5	0.3\\
0.3	0.6	0.3\\
0.3	0.7	0.3\\
0.3	0.8	0.3\\
0.3	0.9	0.3\\
0.3	1	0.3\\
0.4	0	0\\
0.4	0.1	0.1\\
0.4	0.2	0.2\\
0.4	0.3	0.3\\
0.4	0.4	0.4\\
0.4	0.5	0.4\\
0.4	0.6	0.4\\
0.4	0.7	0.4\\
0.4	0.8	0.4\\
0.4	0.9	0.4\\
0.4	1	0.4\\
0.5	0	0\\
0.5	0.1	0.1\\
0.5	0.2	0.2\\
0.5	0.3	0.3\\
0.5	0.4	0.4\\
0.5	0.5	0.5\\
0.5	0.6	0.5\\
0.5	0.7	0.5\\
0.5	0.8	0.5\\
0.5	0.9	0.5\\
0.5	1	0.5\\
0.6	0	0\\
0.6	0.1	0.1\\
0.6	0.2	0.2\\
0.6	0.3	0.3\\
0.6	0.4	0.4\\
0.6	0.5	0.5\\
0.6	0.6	0.6\\
0.6	0.7	0.6\\
0.6	0.8	0.6\\
0.6	0.9	0.6\\
0.6	1	0.6\\
0.7	0	0\\
0.7	0.1	0.1\\
0.7	0.2	0.2\\
0.7	0.3	0.3\\
0.7	0.4	0.4\\
0.7	0.5	0.5\\
0.7	0.6	0.6\\
0.7	0.7	0.7\\
0.7	0.8	0.7\\
0.7	0.9	0.7\\
0.7	1	0.7\\
0.8	0	0\\
0.8	0.1	0.1\\
0.8	0.2	0.2\\
0.8	0.3	0.3\\
0.8	0.4	0.4\\
0.8	0.5	0.5\\
0.8	0.6	0.6\\
0.8	0.7	0.7\\
0.8	0.8	0.8\\
0.8	0.9	0.8\\
0.8	1	0.8\\
0.9	0	0\\
0.9	0.1	0.1\\
0.9	0.2	0.2\\
0.9	0.3	0.3\\
0.9	0.4	0.4\\
0.9	0.5	0.5\\
0.9	0.6	0.6\\
0.9	0.7	0.7\\
0.9	0.8	0.8\\
0.9	0.9	0.9\\
0.9	1	0.9\\
1	0	0\\
1	0.1	0.1\\
1	0.2	0.2\\
1	0.3	0.3\\
1	0.4	0.4\\
1	0.5	0.5\\
1	0.6	0.6\\
1	0.7	0.7\\
1	0.8	0.8\\
1	0.9	0.9\\
1	1	1\\
};
\end{axis}
\end{tikzpicture}%
	\end{subfigure}
	\qquad
	\begin{subfigure}{0.45\textwidth}
		\caption{$G_{O}(x,y) = \sqrt{x \cdot y}$}
		\setlength\figureheight{3cm}
		\setlength\figurewidth{4cm}
		% This file was created by matlab2tikz v0.4.7 (commit 921463bb921f990ffc6a7f597361910de95829d7) running on MATLAB 8.0.
% Copyright (c) 2008--2014, Nico Schlömer <nico.schloemer@gmail.com>
% All rights reserved.
% Minimal pgfplots version: 1.3
% 
\begin{tikzpicture}

\begin{axis}[%
width=\figurewidth,
height=\figureheight,
view={-37.5}{30},
scale only axis,
xmin=0,
xmax=1,
xlabel={x},
xmajorgrids,
ymin=0,
ymax=1,
ylabel={y},
ymajorgrids,
zmin=0,
zmax=1,
zlabel={$\text{G}_{\text{O}}\text{(x,y)}$},
zmajorgrids,
axis x line*=bottom,
axis y line*=left,
axis z line*=left
]

\addplot3[%
surf,
shader=faceted,
draw=black,
colormap/jet,
mesh/rows=11]
table[row sep=crcr,header=false] {0	0	0\\
0	0.1	0\\
0	0.2	0\\
0	0.3	0\\
0	0.4	0\\
0	0.5	0\\
0	0.6	0\\
0	0.7	0\\
0	0.8	0\\
0	0.9	0\\
0	1	0\\
0.1	0	0\\
0.1	0.1	0.1\\
0.1	0.2	0.14142135623731\\
0.1	0.3	0.173205080756888\\
0.1	0.4	0.2\\
0.1	0.5	0.223606797749979\\
0.1	0.6	0.244948974278318\\
0.1	0.7	0.264575131106459\\
0.1	0.8	0.282842712474619\\
0.1	0.9	0.3\\
0.1	1	0.316227766016838\\
0.2	0	0\\
0.2	0.1	0.14142135623731\\
0.2	0.2	0.2\\
0.2	0.3	0.244948974278318\\
0.2	0.4	0.282842712474619\\
0.2	0.5	0.316227766016838\\
0.2	0.6	0.346410161513775\\
0.2	0.7	0.374165738677394\\
0.2	0.8	0.4\\
0.2	0.9	0.424264068711929\\
0.2	1	0.447213595499958\\
0.3	0	0\\
0.3	0.1	0.173205080756888\\
0.3	0.2	0.244948974278318\\
0.3	0.3	0.3\\
0.3	0.4	0.346410161513776\\
0.3	0.5	0.387298334620742\\
0.3	0.6	0.424264068711929\\
0.3	0.7	0.458257569495584\\
0.3	0.8	0.489897948556636\\
0.3	0.9	0.519615242270663\\
0.3	1	0.547722557505166\\
0.4	0	0\\
0.4	0.1	0.2\\
0.4	0.2	0.282842712474619\\
0.4	0.3	0.346410161513776\\
0.4	0.4	0.4\\
0.4	0.5	0.447213595499958\\
0.4	0.6	0.489897948556636\\
0.4	0.7	0.529150262212918\\
0.4	0.8	0.565685424949238\\
0.4	0.9	0.6\\
0.4	1	0.632455532033676\\
0.5	0	0\\
0.5	0.1	0.223606797749979\\
0.5	0.2	0.316227766016838\\
0.5	0.3	0.387298334620742\\
0.5	0.4	0.447213595499958\\
0.5	0.5	0.5\\
0.5	0.6	0.547722557505166\\
0.5	0.7	0.591607978309962\\
0.5	0.8	0.632455532033676\\
0.5	0.9	0.670820393249937\\
0.5	1	0.707106781186548\\
0.6	0	0\\
0.6	0.1	0.244948974278318\\
0.6	0.2	0.346410161513775\\
0.6	0.3	0.424264068711929\\
0.6	0.4	0.489897948556636\\
0.6	0.5	0.547722557505166\\
0.6	0.6	0.6\\
0.6	0.7	0.648074069840786\\
0.6	0.8	0.692820323027551\\
0.6	0.9	0.734846922834953\\
0.6	1	0.774596669241483\\
0.7	0	0\\
0.7	0.1	0.264575131106459\\
0.7	0.2	0.374165738677394\\
0.7	0.3	0.458257569495584\\
0.7	0.4	0.529150262212918\\
0.7	0.5	0.591607978309962\\
0.7	0.6	0.648074069840786\\
0.7	0.7	0.7\\
0.7	0.8	0.748331477354788\\
0.7	0.9	0.793725393319377\\
0.7	1	0.836660026534076\\
0.8	0	0\\
0.8	0.1	0.282842712474619\\
0.8	0.2	0.4\\
0.8	0.3	0.489897948556636\\
0.8	0.4	0.565685424949238\\
0.8	0.5	0.632455532033676\\
0.8	0.6	0.692820323027551\\
0.8	0.7	0.748331477354788\\
0.8	0.8	0.8\\
0.8	0.9	0.848528137423857\\
0.8	1	0.894427190999916\\
0.9	0	0\\
0.9	0.1	0.3\\
0.9	0.2	0.424264068711929\\
0.9	0.3	0.519615242270663\\
0.9	0.4	0.6\\
0.9	0.5	0.670820393249937\\
0.9	0.6	0.734846922834953\\
0.9	0.7	0.793725393319377\\
0.9	0.8	0.848528137423857\\
0.9	0.9	0.9\\
0.9	1	0.948683298050514\\
1	0	0\\
1	0.1	0.316227766016838\\
1	0.2	0.447213595499958\\
1	0.3	0.547722557505166\\
1	0.4	0.632455532033676\\
1	0.5	0.707106781186548\\
1	0.6	0.774596669241483\\
1	0.7	0.836660026534076\\
1	0.8	0.894427190999916\\
1	0.9	0.948683298050514\\
1	1	1\\
};
\end{axis}
\end{tikzpicture}%
	\end{subfigure}
	\end{figure}
\end{frame}

\begin{frame}{Funciones de solapamiento (cont.)}
	\begin{block}{Teorema: Construcción de funciones de solapamiento}
		Una función $G_{O} : [0,1]^{2} \rightarrow [0,1]$ es una función de solapamiento si y sólo si puede ser expresada como:
		\begin{equation}
		G_{O}(x,y) = \frac{f(x,y)}{f(x,y) + h(x,y)}
		\end{equation}
		para cualesquiera $f,h: [0,1]^{2} \rightarrow [0,1]$ tales que:
		\begin{enumerate}
		   \item f y h son simétricas;
		   \item f es no decreciente y h es no creciente;
		   \item $f(x,y) = 0$ si y sólo si $min(x.y) = 0$;
		   \item $h(x,y) = 0$ si y sólo si $min(x.y) = 1$;
		   \item f y h son continuas
		\end{enumerate}
	\end{block}
\end{frame}

\begin{frame}{Funciones de solapamiento (cont.)}
	Algunos ejemplos de construcción de funciones de solapamiento: 
	\vspace{0.5cm}
	\begin{enumerate}
		\item Si $f(x,y)=min(x,y)$ y $h(x,y) = max(1-x,1-y)$ entonces: 
		\begin{equation}
		G_{O}(x,y) = min(x,y)
		\end{equation}
		\item Si $f(x,y)=\sqrt{x \cdot y}$ y $h(x,y) = max(1-x,1-y)$ entonces: 
		\begin{equation}G_{O}(x,y) = \frac{\sqrt{x \cdot y}}{\sqrt{x \cdot y} + max(1-x,1-y)}
		\end{equation}
		\item Si $f(x,y)=\sqrt{x \cdot y}$ y $h(x,y) = 1 - x \cdot y $ entonces: 
		\begin{equation}G_{O}(x,y) = \frac{\sqrt{x \cdot y}}{\sqrt{x \cdot y} + 1 - x \cdot y}
		\end{equation}
	\end{enumerate}
\end{frame}

\subsection{Índices de solapamiento}
\begin{frame}{Índices de solapamiento}
	\begin{block}{Definición: índice de solapamiento}
		Un índice de solapamiento es una función $O : FS(U) \times FS(U) \rightarrow [0,1]$ tal que:
		\begin{enumerate}
		\item $O(A,B) = 0$ si y sólo si A y B tienen soportes disjuntos, es decir: 
			\begin{equation}
				A(u_{i}) \cdot B(u_{i}) = 0 \qquad \forall u_{i} \in U
			\end{equation}
		\item $O(A,B) = O(B,A)$\label{BO2}
		\item Si $B \leq C$, entonces $O(A,B) \leq O(A,C)$\label{BO3}
		\end{enumerate}
	\end{block}
	\begin{itemize}
		\item Los índices de solapamiento proporcionan una medida de similaridad entre dos conjuntos difusos.
		\item Muy útiles para determinar ``cómo de parecidos'' son dos conjuntos difusos.
	\end{itemize}
\end{frame}

\begin{frame}{Índices de solapamiento (cont.)}
	\begin{block}{Teorema: construcción de índices de solapamiento}
		\begin{enumerate}
			\item Sea $M : [0,1]^{2} \rightarrow [0,1]$ una función de agregación tal que $M(x_{1},\cdots,x_{n}) = 0$ si y sólo si $x_{1} = \cdots = x_{n} = 0$
			\item Sea $G_{O} : [0,1]^{2} \rightarrow [0,1]$ una función de solapamiento.
		\end{enumerate}
		Entonces, la función $O: F(U) \times F(U) \rightarrow [0,1]$ definida como:
		\begin{equation}
		O(A,B) = M(G_{O}(A(u_{1}),B(u_{1})),\cdots,G_{O}(A(u_{n}),B(u_{n})))
		\end{equation}
		es un índice de solapamiento.
	\end{block}
\end{frame}

\begin{frame}{Índices de solapamiento (cont.)}
Algunos ejemplos de construcción de índices de solapamiento:
\begin{itemize}
	\item Si $M = \text{Media aritmética}$ y $G_O = x \cdot y$:
	\begin{equation}
	O_{\pi}(A,B) = \frac{1}{n}\sum_{i=1}^{n}\mu_A(x_i) \cdot \mu_B(x_i)
	\end{equation}
	\item Si $M = \max(x_1,\cdots,x_n)$ y $G_O = \min(x,y)$:
	\begin{equation}
	O_{Z}(A,B) = \max\limits_{i=1}^{n}(\min(\mu_A(x_i),\mu_B(x_i)))
	\end{equation}
	%\item Si $M = \text{Media aritmética}$ y $G_O = \sqrt{x \cdot y}$:
	%\begin{equation}
	%O_{\sqrt{\text{ }}}(A,B) =  \frac{1}{n}\sum_{i=1}^{n}\sqrt{\mu_A(x_i)*\mu_B(x_i)}
	%\end{equation}
	\item Si $M = \text{Media aritmética}$ y $G_O = \sin(\frac{\pi}{2}(x \cdot y))^{\frac{1}{4}}$:
	\begin{equation}
	O_{sin}(A, B) = \frac{1}{n}\sum_{i=1}^{n}\sin(\frac{\pi}{2}(\mu_A(x_i) \cdot \mu_B(x_i)))^{\frac{1}{4}}
	\end{equation}
\end{itemize}
\end{frame}

\section{Lógica difusa}

\subsection{Método de Mamdani}

\begin{frame}{Método de Mamdani}
Propuesto por Mamdani y Assilian en 1975 para realizar el control de un motor de vapor a partir de un conjunto de reglas obtenidas de operadores humanos experimentados:

\begin{block}{Algoritmo: Método de Mamdani}
	\scalebox{0.90}{
		\begin{algorithm}[H]
		\DontPrintSemicolon
		\KwIn{Un conjunto de reglas $R_{i}$ ($i \in \{1,\ldots,n\}$) con varios antecedentes $A_{ij}$ ($j \in \{1,\ldots,m\}$)  y entradas escalares $x_1,\ldots,x_m$.}
		\KwOut{\emph{B'}.}
		\For{$i \in \{1,\ldots,n\}$} {
		Calcular $k_i = \min(\mu_{Ai1}(x_1),\ldots,\mu_{Aim}(x_m))$ \\
		Calcular $B_i' = \{(y,\min(B_i(y),k_i))|y \in Y\}$
		}
		Construir $B' = \{(y, B'(y))|y \in Y\}$ dado por: \\
		\centering
		\nonl $B'(y) = \max\limits_{i=1}^{n}(B_i')$.\\
		\Return{$B'$}\;
		\end{algorithm}
	}
\end{block}
\end{frame}

\subsection{Método de interpolación}
\begin{frame}{Método de interpolación}
\begin{itemize}
\item Históricamente uno de los métodos más utilizados para resolver sistemas basados en reglas ha sido el \emph{método de interpolación}.
\item  En este método se utiliza la consistencia de Zadeh $O_{Z}(A,B) = \max\limits_{i=1}^{n}(\min(\mu_A(x_i),\mu_B(x_i)))$.
\end{itemize}
\begin{block}{Algoritmo: Método de interpolación}
	\scalebox{0.90}{
		\begin{algorithm}[H]
		\DontPrintSemicolon
		\KwIn{Un conjunto de reglas $R_{j}$, con $j \in \{1,\ldots,n\}$, un hecho \emph{A'} y el índice de consistencia $O_{Z}$}
		\KwOut{\emph{B'}.}
		\For{$j \in \{1,\ldots,n\}$} {
		Calcular $O_{Z}(A',A_{j}) = \max\limits_{x \in X}(\min(A'(x),A_{j}(x))) $
		}
		Construir $B' = \{(y, B'(y))|y \in Y\}$ dado por: \\
		\centering
		\nonl $B'(y) = \max\limits_{j=1}^{n}(\min(B_{j}(y),O_{Z}(A', A_{j})))$.\\
		\Return{$B'$}\;
		\end{algorithm}
	}
\end{block}
\end{frame}

\begin{frame}{Método de interpolación (cont.)}
El método de interpolación se puede generalizar para utilizar cualquier índice de solapamiento y reglas con varios antecedentes:
\begin{block}{Método de interpolación basado en índices de solapamiento}
	\scalebox{0.80}{
		\begin{algorithm}[H]
		\DontPrintSemicolon
		\KwIn{Un conjunto de reglas $R_{j}$ con varios antecedentes, con $j \in \{1,\ldots,n\}$ y un hecho \emph{A'}.}
		\KwOut{\emph{B'}.}
		Seleccionar un operador de agregación $M$, una t-norma \emph{T} y un índice de solapamiento \emph{O}.\\
		\For{$i =1 \to n$} {
		Calcular $O(A'_{i}, A_{i1}),\ldots,O(A'_{m}, A_{im})$\\
		Calcular $k_{i} = T(O(A'_{i}, A_{i1}),\ldots,O(A'_{m}, A_{im}))$\\
		Construir sobre el universo de referencia \emph{Y} el conjunto $K_{i} = \{(y,k_{i})|y \in Y\}$
		}
		Construir $B' = \{(y, B'(y))|y \in Y\}$ dado por: \\
		\centering
		\nonl $B'(y) = \overset{n}{\underset{i=1}{M}}(\min(K_{i},B_{i}))$.\\
		\Return{$B'$}\;
		\end{algorithm}
	}
\end{block}
\end{frame}

\section{Aplicación práctica: riesgo de incendios forestales}
\begin{frame}{Riesgo de incendios forestales}
Entradas (medidas por la red de sensores):
\begin{enumerate}
   \item \bfseries $\chi_1$ - Temperatura: \normalfont medida en grados centígrados (0ºC a 120ºC).
   \item \bfseries $\chi_2$ - Humo: \normalfont medida en partes por millón (0 a 100ppm).
   \item \bfseries $\chi_3$ - Luz: \normalfont medida en lux (0 a 1000 lux).
   \item \bfseries $\chi_4$ - Humedad: \normalfont medida en partes por millón (0 a 100ppm).
   \item \bfseries $\chi_5$ - Distancia: \normalfont medida en metros (0 a 80m).
\end{enumerate}
Salida:
\begin{enumerate}
   \item \bfseries $y$ - Riesgo de incendio: \normalfont porcentaje (0-100\%).
\end{enumerate}
\end{frame}

\begin{frame}
\begin{figure}[H]
	\centering
	\begin{subfigure}[b]{0.45\textwidth}
		\setlength\figureheight{1.35cm}
		\setlength\figurewidth{4.5cm}
		% This file was created by matlab2tikz v0.4.7 (commit 24ad43488791a40a95315e042d4ef2f4890dac4f) running on MATLAB 8.0.
% Copyright (c) 2008--2014, Nico Schlömer <nico.schloemer@gmail.com>
% All rights reserved.
% Minimal pgfplots version: 1.3
% 
\begin{tikzpicture}

\begin{axis}[%
width=\figurewidth,
height=\figureheight,
clip=false,
scale only axis,
xmin=0,
xmax=120,
ymin=0,
ymax=1
]
\addplot [color=black,solid,line width=1.4pt,forget plot]
  table[row sep=crcr]{0	1\\
1	0.98\\
2	0.96\\
3	0.94\\
4	0.92\\
5	0.9\\
6	0.88\\
7	0.86\\
8	0.84\\
9	0.82\\
10	0.8\\
11	0.78\\
12	0.76\\
13	0.74\\
14	0.72\\
15	0.7\\
16	0.68\\
17	0.66\\
18	0.64\\
19	0.62\\
20	0.6\\
21	0.58\\
22	0.56\\
23	0.54\\
24	0.52\\
25	0.5\\
26	0.48\\
27	0.46\\
28	0.44\\
29	0.42\\
30	0.4\\
31	0.38\\
32	0.36\\
33	0.34\\
34	0.32\\
35	0.3\\
36	0.28\\
37	0.26\\
38	0.24\\
39	0.22\\
40	0.2\\
41	0.18\\
42	0.16\\
43	0.14\\
44	0.12\\
45	0.1\\
46	0.08\\
47	0.0600000000000001\\
48	0.04\\
49	0.02\\
50	0\\
51	0\\
52	0\\
53	0\\
54	0\\
55	0\\
56	0\\
57	0\\
58	0\\
59	0\\
60	0\\
61	0\\
62	0\\
63	0\\
64	0\\
65	0\\
66	0\\
67	0\\
68	0\\
69	0\\
70	0\\
71	0\\
72	0\\
73	0\\
74	0\\
75	0\\
76	0\\
77	0\\
78	0\\
79	0\\
80	0\\
81	0\\
82	0\\
83	0\\
84	0\\
85	0\\
86	0\\
87	0\\
88	0\\
89	0\\
90	0\\
91	0\\
92	0\\
93	0\\
94	0\\
95	0\\
96	0\\
97	0\\
98	0\\
99	0\\
100	0\\
101	0\\
102	0\\
103	0\\
104	0\\
105	0\\
106	0\\
107	0\\
108	0\\
109	0\\
110	0\\
111	0\\
112	0\\
113	0\\
114	0\\
115	0\\
116	0\\
117	0\\
118	0\\
119	0\\
120	0\\
};
\addplot [color=black,solid,line width=1.4pt,forget plot]
  table[row sep=crcr]{0	0\\
1	0\\
2	0\\
3	0\\
4	0\\
5	0\\
6	0\\
7	0\\
8	0\\
9	0\\
10	0\\
11	0.02\\
12	0.04\\
13	0.06\\
14	0.08\\
15	0.1\\
16	0.12\\
17	0.14\\
18	0.16\\
19	0.18\\
20	0.2\\
21	0.22\\
22	0.24\\
23	0.26\\
24	0.28\\
25	0.3\\
26	0.32\\
27	0.34\\
28	0.36\\
29	0.38\\
30	0.4\\
31	0.42\\
32	0.44\\
33	0.46\\
34	0.48\\
35	0.5\\
36	0.52\\
37	0.54\\
38	0.56\\
39	0.58\\
40	0.6\\
41	0.62\\
42	0.64\\
43	0.66\\
44	0.68\\
45	0.7\\
46	0.72\\
47	0.74\\
48	0.76\\
49	0.78\\
50	0.8\\
51	0.82\\
52	0.84\\
53	0.86\\
54	0.88\\
55	0.9\\
56	0.92\\
57	0.94\\
58	0.96\\
59	0.98\\
60	1\\
61	0.98\\
62	0.96\\
63	0.94\\
64	0.92\\
65	0.9\\
66	0.88\\
67	0.86\\
68	0.84\\
69	0.82\\
70	0.8\\
71	0.78\\
72	0.76\\
73	0.74\\
74	0.72\\
75	0.7\\
76	0.68\\
77	0.66\\
78	0.64\\
79	0.62\\
80	0.6\\
81	0.58\\
82	0.56\\
83	0.54\\
84	0.52\\
85	0.5\\
86	0.48\\
87	0.46\\
88	0.44\\
89	0.42\\
90	0.4\\
91	0.38\\
92	0.36\\
93	0.34\\
94	0.32\\
95	0.3\\
96	0.28\\
97	0.26\\
98	0.24\\
99	0.22\\
100	0.2\\
101	0.18\\
102	0.16\\
103	0.14\\
104	0.12\\
105	0.1\\
106	0.08\\
107	0.0600000000000001\\
108	0.04\\
109	0.02\\
110	0\\
111	0\\
112	0\\
113	0\\
114	0\\
115	0\\
116	0\\
117	0\\
118	0\\
119	0\\
120	0\\
};
\addplot [color=black,solid,line width=1.4pt,forget plot]
  table[row sep=crcr]{0	0\\
1	0\\
2	0\\
3	0\\
4	0\\
5	0\\
6	0\\
7	0\\
8	0\\
9	0\\
10	0\\
11	0\\
12	0\\
13	0\\
14	0\\
15	0\\
16	0\\
17	0\\
18	0\\
19	0\\
20	0\\
21	0\\
22	0\\
23	0\\
24	0\\
25	0\\
26	0\\
27	0\\
28	0\\
29	0\\
30	0\\
31	0\\
32	0\\
33	0\\
34	0\\
35	0\\
36	0\\
37	0\\
38	0\\
39	0\\
40	0\\
41	0\\
42	0\\
43	0\\
44	0\\
45	0\\
46	0\\
47	0\\
48	0\\
49	0\\
50	0\\
51	0\\
52	0\\
53	0\\
54	0\\
55	0\\
56	0\\
57	0\\
58	0\\
59	0\\
60	0\\
61	0\\
62	0\\
63	0\\
64	0\\
65	0\\
66	0\\
67	0\\
68	0\\
69	0\\
70	0\\
71	0.02\\
72	0.04\\
73	0.06\\
74	0.08\\
75	0.1\\
76	0.12\\
77	0.14\\
78	0.16\\
79	0.18\\
80	0.2\\
81	0.22\\
82	0.24\\
83	0.26\\
84	0.28\\
85	0.3\\
86	0.32\\
87	0.34\\
88	0.36\\
89	0.38\\
90	0.4\\
91	0.42\\
92	0.44\\
93	0.46\\
94	0.48\\
95	0.5\\
96	0.52\\
97	0.54\\
98	0.56\\
99	0.58\\
100	0.6\\
101	0.62\\
102	0.64\\
103	0.66\\
104	0.68\\
105	0.7\\
106	0.72\\
107	0.74\\
108	0.76\\
109	0.78\\
110	0.8\\
111	0.82\\
112	0.84\\
113	0.86\\
114	0.88\\
115	0.9\\
116	0.92\\
117	0.94\\
118	0.96\\
119	0.98\\
120	1\\
};
\node[right, inner sep=0mm, text=black]
at (axis cs:1,1.15,0) {L};
\node[right, inner sep=0mm, text=black]
at (axis cs:55,1.15,0) {M};
\node[right, inner sep=0mm, text=black]
at (axis cs:105,1.15,0) {H};
\end{axis}
\end{tikzpicture}%
		\caption{$\chi_1$ - Temperatura (ºC)}
		\label{fig:temp-lang-variable}
		\vspace*{2mm}
	\end{subfigure}
	\qquad
	\begin{subfigure}[b]{0.45\textwidth}
		\setlength\figureheight{1.35cm}
		\setlength\figurewidth{4.5cm}
		% This file was created by matlab2tikz v0.4.7 (commit 24ad43488791a40a95315e042d4ef2f4890dac4f) running on MATLAB 8.0.
% Copyright (c) 2008--2014, Nico Schlömer <nico.schloemer@gmail.com>
% All rights reserved.
% Minimal pgfplots version: 1.3
% 
\begin{tikzpicture}

\begin{axis}[%
width=\figurewidth,
height=\figureheight,
clip=false,
scale only axis,
xmin=0,
xmax=100,
ymin=0,
ymax=1
]
\addplot [color=black,solid,line width=1.4pt,forget plot]
  table[row sep=crcr]{0	1\\
1	0.975\\
2	0.95\\
3	0.925\\
4	0.9\\
5	0.875\\
6	0.85\\
7	0.825\\
8	0.8\\
9	0.775\\
10	0.75\\
11	0.725\\
12	0.7\\
13	0.675\\
14	0.65\\
15	0.625\\
16	0.6\\
17	0.575\\
18	0.55\\
19	0.525\\
20	0.5\\
21	0.475\\
22	0.45\\
23	0.425\\
24	0.4\\
25	0.375\\
26	0.35\\
27	0.325\\
28	0.3\\
29	0.275\\
30	0.25\\
31	0.225\\
32	0.2\\
33	0.175\\
34	0.15\\
35	0.125\\
36	0.1\\
37	0.075\\
38	0.05\\
39	0.025\\
40	0\\
41	0\\
42	0\\
43	0\\
44	0\\
45	0\\
46	0\\
47	0\\
48	0\\
49	0\\
50	0\\
51	0\\
52	0\\
53	0\\
54	0\\
55	0\\
56	0\\
57	0\\
58	0\\
59	0\\
60	0\\
61	0\\
62	0\\
63	0\\
64	0\\
65	0\\
66	0\\
67	0\\
68	0\\
69	0\\
70	0\\
71	0\\
72	0\\
73	0\\
74	0\\
75	0\\
76	0\\
77	0\\
78	0\\
79	0\\
80	0\\
81	0\\
82	0\\
83	0\\
84	0\\
85	0\\
86	0\\
87	0\\
88	0\\
89	0\\
90	0\\
91	0\\
92	0\\
93	0\\
94	0\\
95	0\\
96	0\\
97	0\\
98	0\\
99	0\\
100	0\\
};
\addplot [color=black,solid,line width=1.4pt,forget plot]
  table[row sep=crcr]{0	0\\
1	0\\
2	0\\
3	0\\
4	0\\
5	0\\
6	0\\
7	0\\
8	0\\
9	0\\
10	0\\
11	0.025\\
12	0.05\\
13	0.075\\
14	0.1\\
15	0.125\\
16	0.15\\
17	0.175\\
18	0.2\\
19	0.225\\
20	0.25\\
21	0.275\\
22	0.3\\
23	0.325\\
24	0.35\\
25	0.375\\
26	0.4\\
27	0.425\\
28	0.45\\
29	0.475\\
30	0.5\\
31	0.525\\
32	0.55\\
33	0.575\\
34	0.6\\
35	0.625\\
36	0.65\\
37	0.675\\
38	0.7\\
39	0.725\\
40	0.75\\
41	0.775\\
42	0.8\\
43	0.825\\
44	0.85\\
45	0.875\\
46	0.9\\
47	0.925\\
48	0.95\\
49	0.975\\
50	1\\
51	0.975\\
52	0.95\\
53	0.925\\
54	0.9\\
55	0.875\\
56	0.85\\
57	0.825\\
58	0.8\\
59	0.775\\
60	0.75\\
61	0.725\\
62	0.7\\
63	0.675\\
64	0.65\\
65	0.625\\
66	0.6\\
67	0.575\\
68	0.55\\
69	0.525\\
70	0.5\\
71	0.475\\
72	0.45\\
73	0.425\\
74	0.4\\
75	0.375\\
76	0.35\\
77	0.325\\
78	0.3\\
79	0.275\\
80	0.25\\
81	0.225\\
82	0.2\\
83	0.175\\
84	0.15\\
85	0.125\\
86	0.1\\
87	0.075\\
88	0.05\\
89	0.025\\
90	0\\
91	0\\
92	0\\
93	0\\
94	0\\
95	0\\
96	0\\
97	0\\
98	0\\
99	0\\
100	0\\
};
\addplot [color=black,solid,line width=1.4pt,forget plot]
  table[row sep=crcr]{0	0\\
1	0\\
2	0\\
3	0\\
4	0\\
5	0\\
6	0\\
7	0\\
8	0\\
9	0\\
10	0\\
11	0\\
12	0\\
13	0\\
14	0\\
15	0\\
16	0\\
17	0\\
18	0\\
19	0\\
20	0\\
21	0\\
22	0\\
23	0\\
24	0\\
25	0\\
26	0\\
27	0\\
28	0\\
29	0\\
30	0\\
31	0\\
32	0\\
33	0\\
34	0\\
35	0\\
36	0\\
37	0\\
38	0\\
39	0\\
40	0\\
41	0\\
42	0\\
43	0\\
44	0\\
45	0\\
46	0\\
47	0\\
48	0\\
49	0\\
50	0\\
51	0\\
52	0\\
53	0\\
54	0\\
55	0\\
56	0\\
57	0\\
58	0\\
59	0\\
60	0\\
61	0.025\\
62	0.05\\
63	0.075\\
64	0.1\\
65	0.125\\
66	0.15\\
67	0.175\\
68	0.2\\
69	0.225\\
70	0.25\\
71	0.275\\
72	0.3\\
73	0.325\\
74	0.35\\
75	0.375\\
76	0.4\\
77	0.425\\
78	0.45\\
79	0.475\\
80	0.5\\
81	0.525\\
82	0.55\\
83	0.575\\
84	0.6\\
85	0.625\\
86	0.65\\
87	0.675\\
88	0.7\\
89	0.725\\
90	0.75\\
91	0.775\\
92	0.8\\
93	0.825\\
94	0.85\\
95	0.875\\
96	0.9\\
97	0.925\\
98	0.95\\
99	0.975\\
100	1\\
};
\node[right, inner sep=0mm, text=black]
at (axis cs:1,1.15,0) {L};
\node[right, inner sep=0mm, text=black]
at (axis cs:45,1.15,0) {M};
\node[right, inner sep=0mm, text=black]
at (axis cs:90,1.15,0) {H};
\end{axis}
\end{tikzpicture}%
		\caption{$\chi_2$ - Humo (ppm)}
		\label{fig:smoke-lang-variable}
		\vspace*{2mm}
	\end{subfigure}
	
	\begin{subfigure}[b]{0.45\textwidth}
		\setlength\figureheight{1.35cm}
		\setlength\figurewidth{4.5cm}
		% This file was created by matlab2tikz v0.4.7 (commit 24ad43488791a40a95315e042d4ef2f4890dac4f) running on MATLAB 8.0.
% Copyright (c) 2008--2014, Nico Schlömer <nico.schloemer@gmail.com>
% All rights reserved.
% Minimal pgfplots version: 1.3
% 
\begin{tikzpicture}

\begin{axis}[%
width=\figurewidth,
height=\figureheight,
clip=false,
scale only axis,
xmin=0,
xmax=1000,
ymin=0,
ymax=1
]
\addplot [color=black,solid,line width=1.4pt,forget plot]
  table[row sep=crcr]{0	1\\
1	0.9975\\
2	0.995\\
3	0.9925\\
4	0.99\\
5	0.9875\\
6	0.985\\
7	0.9825\\
8	0.98\\
9	0.9775\\
10	0.975\\
11	0.9725\\
12	0.97\\
13	0.9675\\
14	0.965\\
15	0.9625\\
16	0.96\\
17	0.9575\\
18	0.955\\
19	0.9525\\
20	0.95\\
21	0.9475\\
22	0.945\\
23	0.9425\\
24	0.94\\
25	0.9375\\
26	0.935\\
27	0.9325\\
28	0.93\\
29	0.9275\\
30	0.925\\
31	0.9225\\
32	0.92\\
33	0.9175\\
34	0.915\\
35	0.9125\\
36	0.91\\
37	0.9075\\
38	0.905\\
39	0.9025\\
40	0.9\\
41	0.8975\\
42	0.895\\
43	0.8925\\
44	0.89\\
45	0.8875\\
46	0.885\\
47	0.8825\\
48	0.88\\
49	0.8775\\
50	0.875\\
51	0.8725\\
52	0.87\\
53	0.8675\\
54	0.865\\
55	0.8625\\
56	0.86\\
57	0.8575\\
58	0.855\\
59	0.8525\\
60	0.85\\
61	0.8475\\
62	0.845\\
63	0.8425\\
64	0.84\\
65	0.8375\\
66	0.835\\
67	0.8325\\
68	0.83\\
69	0.8275\\
70	0.825\\
71	0.8225\\
72	0.82\\
73	0.8175\\
74	0.815\\
75	0.8125\\
76	0.81\\
77	0.8075\\
78	0.805\\
79	0.8025\\
80	0.8\\
81	0.7975\\
82	0.795\\
83	0.7925\\
84	0.79\\
85	0.7875\\
86	0.785\\
87	0.7825\\
88	0.78\\
89	0.7775\\
90	0.775\\
91	0.7725\\
92	0.77\\
93	0.7675\\
94	0.765\\
95	0.7625\\
96	0.76\\
97	0.7575\\
98	0.755\\
99	0.7525\\
100	0.75\\
101	0.7475\\
102	0.745\\
103	0.7425\\
104	0.74\\
105	0.7375\\
106	0.735\\
107	0.7325\\
108	0.73\\
109	0.7275\\
110	0.725\\
111	0.7225\\
112	0.72\\
113	0.7175\\
114	0.715\\
115	0.7125\\
116	0.71\\
117	0.7075\\
118	0.705\\
119	0.7025\\
120	0.7\\
121	0.6975\\
122	0.695\\
123	0.6925\\
124	0.69\\
125	0.6875\\
126	0.685\\
127	0.6825\\
128	0.68\\
129	0.6775\\
130	0.675\\
131	0.6725\\
132	0.67\\
133	0.6675\\
134	0.665\\
135	0.6625\\
136	0.66\\
137	0.6575\\
138	0.655\\
139	0.6525\\
140	0.65\\
141	0.6475\\
142	0.645\\
143	0.6425\\
144	0.64\\
145	0.6375\\
146	0.635\\
147	0.6325\\
148	0.63\\
149	0.6275\\
150	0.625\\
151	0.6225\\
152	0.62\\
153	0.6175\\
154	0.615\\
155	0.6125\\
156	0.61\\
157	0.6075\\
158	0.605\\
159	0.6025\\
160	0.6\\
161	0.5975\\
162	0.595\\
163	0.5925\\
164	0.59\\
165	0.5875\\
166	0.585\\
167	0.5825\\
168	0.58\\
169	0.5775\\
170	0.575\\
171	0.5725\\
172	0.57\\
173	0.5675\\
174	0.565\\
175	0.5625\\
176	0.56\\
177	0.5575\\
178	0.555\\
179	0.5525\\
180	0.55\\
181	0.5475\\
182	0.545\\
183	0.5425\\
184	0.54\\
185	0.5375\\
186	0.535\\
187	0.5325\\
188	0.53\\
189	0.5275\\
190	0.525\\
191	0.5225\\
192	0.52\\
193	0.5175\\
194	0.515\\
195	0.5125\\
196	0.51\\
197	0.5075\\
198	0.505\\
199	0.5025\\
200	0.5\\
201	0.4975\\
202	0.495\\
203	0.4925\\
204	0.49\\
205	0.4875\\
206	0.485\\
207	0.4825\\
208	0.48\\
209	0.4775\\
210	0.475\\
211	0.4725\\
212	0.47\\
213	0.4675\\
214	0.465\\
215	0.4625\\
216	0.46\\
217	0.4575\\
218	0.455\\
219	0.4525\\
220	0.45\\
221	0.4475\\
222	0.445\\
223	0.4425\\
224	0.44\\
225	0.4375\\
226	0.435\\
227	0.4325\\
228	0.43\\
229	0.4275\\
230	0.425\\
231	0.4225\\
232	0.42\\
233	0.4175\\
234	0.415\\
235	0.4125\\
236	0.41\\
237	0.4075\\
238	0.405\\
239	0.4025\\
240	0.4\\
241	0.3975\\
242	0.395\\
243	0.3925\\
244	0.39\\
245	0.3875\\
246	0.385\\
247	0.3825\\
248	0.38\\
249	0.3775\\
250	0.375\\
251	0.3725\\
252	0.37\\
253	0.3675\\
254	0.365\\
255	0.3625\\
256	0.36\\
257	0.3575\\
258	0.355\\
259	0.3525\\
260	0.35\\
261	0.3475\\
262	0.345\\
263	0.3425\\
264	0.34\\
265	0.3375\\
266	0.335\\
267	0.3325\\
268	0.33\\
269	0.3275\\
270	0.325\\
271	0.3225\\
272	0.32\\
273	0.3175\\
274	0.315\\
275	0.3125\\
276	0.31\\
277	0.3075\\
278	0.305\\
279	0.3025\\
280	0.3\\
281	0.2975\\
282	0.295\\
283	0.2925\\
284	0.29\\
285	0.2875\\
286	0.285\\
287	0.2825\\
288	0.28\\
289	0.2775\\
290	0.275\\
291	0.2725\\
292	0.27\\
293	0.2675\\
294	0.265\\
295	0.2625\\
296	0.26\\
297	0.2575\\
298	0.255\\
299	0.2525\\
300	0.25\\
301	0.2475\\
302	0.245\\
303	0.2425\\
304	0.24\\
305	0.2375\\
306	0.235\\
307	0.2325\\
308	0.23\\
309	0.2275\\
310	0.225\\
311	0.2225\\
312	0.22\\
313	0.2175\\
314	0.215\\
315	0.2125\\
316	0.21\\
317	0.2075\\
318	0.205\\
319	0.2025\\
320	0.2\\
321	0.1975\\
322	0.195\\
323	0.1925\\
324	0.19\\
325	0.1875\\
326	0.185\\
327	0.1825\\
328	0.18\\
329	0.1775\\
330	0.175\\
331	0.1725\\
332	0.17\\
333	0.1675\\
334	0.165\\
335	0.1625\\
336	0.16\\
337	0.1575\\
338	0.155\\
339	0.1525\\
340	0.15\\
341	0.1475\\
342	0.145\\
343	0.1425\\
344	0.14\\
345	0.1375\\
346	0.135\\
347	0.1325\\
348	0.13\\
349	0.1275\\
350	0.125\\
351	0.1225\\
352	0.12\\
353	0.1175\\
354	0.115\\
355	0.1125\\
356	0.11\\
357	0.1075\\
358	0.105\\
359	0.1025\\
360	0.1\\
361	0.0975\\
362	0.095\\
363	0.0925\\
364	0.09\\
365	0.0875\\
366	0.085\\
367	0.0825\\
368	0.08\\
369	0.0775\\
370	0.075\\
371	0.0725\\
372	0.07\\
373	0.0675\\
374	0.0649999999999999\\
375	0.0625\\
376	0.0600000000000001\\
377	0.0575\\
378	0.055\\
379	0.0525\\
380	0.05\\
381	0.0475\\
382	0.045\\
383	0.0425\\
384	0.04\\
385	0.0375\\
386	0.035\\
387	0.0325\\
388	0.03\\
389	0.0275\\
390	0.025\\
391	0.0225\\
392	0.02\\
393	0.0175\\
394	0.015\\
395	0.0125\\
396	0.01\\
397	0.00749999999999995\\
398	0.005\\
399	0.00249999999999995\\
400	0\\
401	0\\
402	0\\
403	0\\
404	0\\
405	0\\
406	0\\
407	0\\
408	0\\
409	0\\
410	0\\
411	0\\
412	0\\
413	0\\
414	0\\
415	0\\
416	0\\
417	0\\
418	0\\
419	0\\
420	0\\
421	0\\
422	0\\
423	0\\
424	0\\
425	0\\
426	0\\
427	0\\
428	0\\
429	0\\
430	0\\
431	0\\
432	0\\
433	0\\
434	0\\
435	0\\
436	0\\
437	0\\
438	0\\
439	0\\
440	0\\
441	0\\
442	0\\
443	0\\
444	0\\
445	0\\
446	0\\
447	0\\
448	0\\
449	0\\
450	0\\
451	0\\
452	0\\
453	0\\
454	0\\
455	0\\
456	0\\
457	0\\
458	0\\
459	0\\
460	0\\
461	0\\
462	0\\
463	0\\
464	0\\
465	0\\
466	0\\
467	0\\
468	0\\
469	0\\
470	0\\
471	0\\
472	0\\
473	0\\
474	0\\
475	0\\
476	0\\
477	0\\
478	0\\
479	0\\
480	0\\
481	0\\
482	0\\
483	0\\
484	0\\
485	0\\
486	0\\
487	0\\
488	0\\
489	0\\
490	0\\
491	0\\
492	0\\
493	0\\
494	0\\
495	0\\
496	0\\
497	0\\
498	0\\
499	0\\
500	0\\
501	0\\
502	0\\
503	0\\
504	0\\
505	0\\
506	0\\
507	0\\
508	0\\
509	0\\
510	0\\
511	0\\
512	0\\
513	0\\
514	0\\
515	0\\
516	0\\
517	0\\
518	0\\
519	0\\
520	0\\
521	0\\
522	0\\
523	0\\
524	0\\
525	0\\
526	0\\
527	0\\
528	0\\
529	0\\
530	0\\
531	0\\
532	0\\
533	0\\
534	0\\
535	0\\
536	0\\
537	0\\
538	0\\
539	0\\
540	0\\
541	0\\
542	0\\
543	0\\
544	0\\
545	0\\
546	0\\
547	0\\
548	0\\
549	0\\
550	0\\
551	0\\
552	0\\
553	0\\
554	0\\
555	0\\
556	0\\
557	0\\
558	0\\
559	0\\
560	0\\
561	0\\
562	0\\
563	0\\
564	0\\
565	0\\
566	0\\
567	0\\
568	0\\
569	0\\
570	0\\
571	0\\
572	0\\
573	0\\
574	0\\
575	0\\
576	0\\
577	0\\
578	0\\
579	0\\
580	0\\
581	0\\
582	0\\
583	0\\
584	0\\
585	0\\
586	0\\
587	0\\
588	0\\
589	0\\
590	0\\
591	0\\
592	0\\
593	0\\
594	0\\
595	0\\
596	0\\
597	0\\
598	0\\
599	0\\
600	0\\
601	0\\
602	0\\
603	0\\
604	0\\
605	0\\
606	0\\
607	0\\
608	0\\
609	0\\
610	0\\
611	0\\
612	0\\
613	0\\
614	0\\
615	0\\
616	0\\
617	0\\
618	0\\
619	0\\
620	0\\
621	0\\
622	0\\
623	0\\
624	0\\
625	0\\
626	0\\
627	0\\
628	0\\
629	0\\
630	0\\
631	0\\
632	0\\
633	0\\
634	0\\
635	0\\
636	0\\
637	0\\
638	0\\
639	0\\
640	0\\
641	0\\
642	0\\
643	0\\
644	0\\
645	0\\
646	0\\
647	0\\
648	0\\
649	0\\
650	0\\
651	0\\
652	0\\
653	0\\
654	0\\
655	0\\
656	0\\
657	0\\
658	0\\
659	0\\
660	0\\
661	0\\
662	0\\
663	0\\
664	0\\
665	0\\
666	0\\
667	0\\
668	0\\
669	0\\
670	0\\
671	0\\
672	0\\
673	0\\
674	0\\
675	0\\
676	0\\
677	0\\
678	0\\
679	0\\
680	0\\
681	0\\
682	0\\
683	0\\
684	0\\
685	0\\
686	0\\
687	0\\
688	0\\
689	0\\
690	0\\
691	0\\
692	0\\
693	0\\
694	0\\
695	0\\
696	0\\
697	0\\
698	0\\
699	0\\
700	0\\
701	0\\
702	0\\
703	0\\
704	0\\
705	0\\
706	0\\
707	0\\
708	0\\
709	0\\
710	0\\
711	0\\
712	0\\
713	0\\
714	0\\
715	0\\
716	0\\
717	0\\
718	0\\
719	0\\
720	0\\
721	0\\
722	0\\
723	0\\
724	0\\
725	0\\
726	0\\
727	0\\
728	0\\
729	0\\
730	0\\
731	0\\
732	0\\
733	0\\
734	0\\
735	0\\
736	0\\
737	0\\
738	0\\
739	0\\
740	0\\
741	0\\
742	0\\
743	0\\
744	0\\
745	0\\
746	0\\
747	0\\
748	0\\
749	0\\
750	0\\
751	0\\
752	0\\
753	0\\
754	0\\
755	0\\
756	0\\
757	0\\
758	0\\
759	0\\
760	0\\
761	0\\
762	0\\
763	0\\
764	0\\
765	0\\
766	0\\
767	0\\
768	0\\
769	0\\
770	0\\
771	0\\
772	0\\
773	0\\
774	0\\
775	0\\
776	0\\
777	0\\
778	0\\
779	0\\
780	0\\
781	0\\
782	0\\
783	0\\
784	0\\
785	0\\
786	0\\
787	0\\
788	0\\
789	0\\
790	0\\
791	0\\
792	0\\
793	0\\
794	0\\
795	0\\
796	0\\
797	0\\
798	0\\
799	0\\
800	0\\
801	0\\
802	0\\
803	0\\
804	0\\
805	0\\
806	0\\
807	0\\
808	0\\
809	0\\
810	0\\
811	0\\
812	0\\
813	0\\
814	0\\
815	0\\
816	0\\
817	0\\
818	0\\
819	0\\
820	0\\
821	0\\
822	0\\
823	0\\
824	0\\
825	0\\
826	0\\
827	0\\
828	0\\
829	0\\
830	0\\
831	0\\
832	0\\
833	0\\
834	0\\
835	0\\
836	0\\
837	0\\
838	0\\
839	0\\
840	0\\
841	0\\
842	0\\
843	0\\
844	0\\
845	0\\
846	0\\
847	0\\
848	0\\
849	0\\
850	0\\
851	0\\
852	0\\
853	0\\
854	0\\
855	0\\
856	0\\
857	0\\
858	0\\
859	0\\
860	0\\
861	0\\
862	0\\
863	0\\
864	0\\
865	0\\
866	0\\
867	0\\
868	0\\
869	0\\
870	0\\
871	0\\
872	0\\
873	0\\
874	0\\
875	0\\
876	0\\
877	0\\
878	0\\
879	0\\
880	0\\
881	0\\
882	0\\
883	0\\
884	0\\
885	0\\
886	0\\
887	0\\
888	0\\
889	0\\
890	0\\
891	0\\
892	0\\
893	0\\
894	0\\
895	0\\
896	0\\
897	0\\
898	0\\
899	0\\
900	0\\
901	0\\
902	0\\
903	0\\
904	0\\
905	0\\
906	0\\
907	0\\
908	0\\
909	0\\
910	0\\
911	0\\
912	0\\
913	0\\
914	0\\
915	0\\
916	0\\
917	0\\
918	0\\
919	0\\
920	0\\
921	0\\
922	0\\
923	0\\
924	0\\
925	0\\
926	0\\
927	0\\
928	0\\
929	0\\
930	0\\
931	0\\
932	0\\
933	0\\
934	0\\
935	0\\
936	0\\
937	0\\
938	0\\
939	0\\
940	0\\
941	0\\
942	0\\
943	0\\
944	0\\
945	0\\
946	0\\
947	0\\
948	0\\
949	0\\
950	0\\
951	0\\
952	0\\
953	0\\
954	0\\
955	0\\
956	0\\
957	0\\
958	0\\
959	0\\
960	0\\
961	0\\
962	0\\
963	0\\
964	0\\
965	0\\
966	0\\
967	0\\
968	0\\
969	0\\
970	0\\
971	0\\
972	0\\
973	0\\
974	0\\
975	0\\
976	0\\
977	0\\
978	0\\
979	0\\
980	0\\
981	0\\
982	0\\
983	0\\
984	0\\
985	0\\
986	0\\
987	0\\
988	0\\
989	0\\
990	0\\
991	0\\
992	0\\
993	0\\
994	0\\
995	0\\
996	0\\
997	0\\
998	0\\
999	0\\
1000	0\\
};
\addplot [color=black,solid,line width=1.4pt,forget plot]
  table[row sep=crcr]{0	0\\
1	0\\
2	0\\
3	0\\
4	0\\
5	0\\
6	0\\
7	0\\
8	0\\
9	0\\
10	0\\
11	0\\
12	0\\
13	0\\
14	0\\
15	0\\
16	0\\
17	0\\
18	0\\
19	0\\
20	0\\
21	0\\
22	0\\
23	0\\
24	0\\
25	0\\
26	0\\
27	0\\
28	0\\
29	0\\
30	0\\
31	0\\
32	0\\
33	0\\
34	0\\
35	0\\
36	0\\
37	0\\
38	0\\
39	0\\
40	0\\
41	0\\
42	0\\
43	0\\
44	0\\
45	0\\
46	0\\
47	0\\
48	0\\
49	0\\
50	0\\
51	0\\
52	0\\
53	0\\
54	0\\
55	0\\
56	0\\
57	0\\
58	0\\
59	0\\
60	0\\
61	0\\
62	0\\
63	0\\
64	0\\
65	0\\
66	0\\
67	0\\
68	0\\
69	0\\
70	0\\
71	0\\
72	0\\
73	0\\
74	0\\
75	0\\
76	0\\
77	0\\
78	0\\
79	0\\
80	0\\
81	0\\
82	0\\
83	0\\
84	0\\
85	0\\
86	0\\
87	0\\
88	0\\
89	0\\
90	0\\
91	0\\
92	0\\
93	0\\
94	0\\
95	0\\
96	0\\
97	0\\
98	0\\
99	0\\
100	0\\
101	0.0025\\
102	0.005\\
103	0.0075\\
104	0.01\\
105	0.0125\\
106	0.015\\
107	0.0175\\
108	0.02\\
109	0.0225\\
110	0.025\\
111	0.0275\\
112	0.03\\
113	0.0325\\
114	0.035\\
115	0.0375\\
116	0.04\\
117	0.0425\\
118	0.045\\
119	0.0475\\
120	0.05\\
121	0.0525\\
122	0.055\\
123	0.0575\\
124	0.06\\
125	0.0625\\
126	0.065\\
127	0.0675\\
128	0.07\\
129	0.0725\\
130	0.075\\
131	0.0775\\
132	0.08\\
133	0.0825\\
134	0.085\\
135	0.0875\\
136	0.09\\
137	0.0925\\
138	0.095\\
139	0.0975\\
140	0.1\\
141	0.1025\\
142	0.105\\
143	0.1075\\
144	0.11\\
145	0.1125\\
146	0.115\\
147	0.1175\\
148	0.12\\
149	0.1225\\
150	0.125\\
151	0.1275\\
152	0.13\\
153	0.1325\\
154	0.135\\
155	0.1375\\
156	0.14\\
157	0.1425\\
158	0.145\\
159	0.1475\\
160	0.15\\
161	0.1525\\
162	0.155\\
163	0.1575\\
164	0.16\\
165	0.1625\\
166	0.165\\
167	0.1675\\
168	0.17\\
169	0.1725\\
170	0.175\\
171	0.1775\\
172	0.18\\
173	0.1825\\
174	0.185\\
175	0.1875\\
176	0.19\\
177	0.1925\\
178	0.195\\
179	0.1975\\
180	0.2\\
181	0.2025\\
182	0.205\\
183	0.2075\\
184	0.21\\
185	0.2125\\
186	0.215\\
187	0.2175\\
188	0.22\\
189	0.2225\\
190	0.225\\
191	0.2275\\
192	0.23\\
193	0.2325\\
194	0.235\\
195	0.2375\\
196	0.24\\
197	0.2425\\
198	0.245\\
199	0.2475\\
200	0.25\\
201	0.2525\\
202	0.255\\
203	0.2575\\
204	0.26\\
205	0.2625\\
206	0.265\\
207	0.2675\\
208	0.27\\
209	0.2725\\
210	0.275\\
211	0.2775\\
212	0.28\\
213	0.2825\\
214	0.285\\
215	0.2875\\
216	0.29\\
217	0.2925\\
218	0.295\\
219	0.2975\\
220	0.3\\
221	0.3025\\
222	0.305\\
223	0.3075\\
224	0.31\\
225	0.3125\\
226	0.315\\
227	0.3175\\
228	0.32\\
229	0.3225\\
230	0.325\\
231	0.3275\\
232	0.33\\
233	0.3325\\
234	0.335\\
235	0.3375\\
236	0.34\\
237	0.3425\\
238	0.345\\
239	0.3475\\
240	0.35\\
241	0.3525\\
242	0.355\\
243	0.3575\\
244	0.36\\
245	0.3625\\
246	0.365\\
247	0.3675\\
248	0.37\\
249	0.3725\\
250	0.375\\
251	0.3775\\
252	0.38\\
253	0.3825\\
254	0.385\\
255	0.3875\\
256	0.39\\
257	0.3925\\
258	0.395\\
259	0.3975\\
260	0.4\\
261	0.4025\\
262	0.405\\
263	0.4075\\
264	0.41\\
265	0.4125\\
266	0.415\\
267	0.4175\\
268	0.42\\
269	0.4225\\
270	0.425\\
271	0.4275\\
272	0.43\\
273	0.4325\\
274	0.435\\
275	0.4375\\
276	0.44\\
277	0.4425\\
278	0.445\\
279	0.4475\\
280	0.45\\
281	0.4525\\
282	0.455\\
283	0.4575\\
284	0.46\\
285	0.4625\\
286	0.465\\
287	0.4675\\
288	0.47\\
289	0.4725\\
290	0.475\\
291	0.4775\\
292	0.48\\
293	0.4825\\
294	0.485\\
295	0.4875\\
296	0.49\\
297	0.4925\\
298	0.495\\
299	0.4975\\
300	0.5\\
301	0.5025\\
302	0.505\\
303	0.5075\\
304	0.51\\
305	0.5125\\
306	0.515\\
307	0.5175\\
308	0.52\\
309	0.5225\\
310	0.525\\
311	0.5275\\
312	0.53\\
313	0.5325\\
314	0.535\\
315	0.5375\\
316	0.54\\
317	0.5425\\
318	0.545\\
319	0.5475\\
320	0.55\\
321	0.5525\\
322	0.555\\
323	0.5575\\
324	0.56\\
325	0.5625\\
326	0.565\\
327	0.5675\\
328	0.57\\
329	0.5725\\
330	0.575\\
331	0.5775\\
332	0.58\\
333	0.5825\\
334	0.585\\
335	0.5875\\
336	0.59\\
337	0.5925\\
338	0.595\\
339	0.5975\\
340	0.6\\
341	0.6025\\
342	0.605\\
343	0.6075\\
344	0.61\\
345	0.6125\\
346	0.615\\
347	0.6175\\
348	0.62\\
349	0.6225\\
350	0.625\\
351	0.6275\\
352	0.63\\
353	0.6325\\
354	0.635\\
355	0.6375\\
356	0.64\\
357	0.6425\\
358	0.645\\
359	0.6475\\
360	0.65\\
361	0.6525\\
362	0.655\\
363	0.6575\\
364	0.66\\
365	0.6625\\
366	0.665\\
367	0.6675\\
368	0.67\\
369	0.6725\\
370	0.675\\
371	0.6775\\
372	0.68\\
373	0.6825\\
374	0.685\\
375	0.6875\\
376	0.69\\
377	0.6925\\
378	0.695\\
379	0.6975\\
380	0.7\\
381	0.7025\\
382	0.705\\
383	0.7075\\
384	0.71\\
385	0.7125\\
386	0.715\\
387	0.7175\\
388	0.72\\
389	0.7225\\
390	0.725\\
391	0.7275\\
392	0.73\\
393	0.7325\\
394	0.735\\
395	0.7375\\
396	0.74\\
397	0.7425\\
398	0.745\\
399	0.7475\\
400	0.75\\
401	0.7525\\
402	0.755\\
403	0.7575\\
404	0.76\\
405	0.7625\\
406	0.765\\
407	0.7675\\
408	0.77\\
409	0.7725\\
410	0.775\\
411	0.7775\\
412	0.78\\
413	0.7825\\
414	0.785\\
415	0.7875\\
416	0.79\\
417	0.7925\\
418	0.795\\
419	0.7975\\
420	0.8\\
421	0.8025\\
422	0.805\\
423	0.8075\\
424	0.81\\
425	0.8125\\
426	0.815\\
427	0.8175\\
428	0.82\\
429	0.8225\\
430	0.825\\
431	0.8275\\
432	0.83\\
433	0.8325\\
434	0.835\\
435	0.8375\\
436	0.84\\
437	0.8425\\
438	0.845\\
439	0.8475\\
440	0.85\\
441	0.8525\\
442	0.855\\
443	0.8575\\
444	0.86\\
445	0.8625\\
446	0.865\\
447	0.8675\\
448	0.87\\
449	0.8725\\
450	0.875\\
451	0.8775\\
452	0.88\\
453	0.8825\\
454	0.885\\
455	0.8875\\
456	0.89\\
457	0.8925\\
458	0.895\\
459	0.8975\\
460	0.9\\
461	0.9025\\
462	0.905\\
463	0.9075\\
464	0.91\\
465	0.9125\\
466	0.915\\
467	0.9175\\
468	0.92\\
469	0.9225\\
470	0.925\\
471	0.9275\\
472	0.93\\
473	0.9325\\
474	0.935\\
475	0.9375\\
476	0.94\\
477	0.9425\\
478	0.945\\
479	0.9475\\
480	0.95\\
481	0.9525\\
482	0.955\\
483	0.9575\\
484	0.96\\
485	0.9625\\
486	0.965\\
487	0.9675\\
488	0.97\\
489	0.9725\\
490	0.975\\
491	0.9775\\
492	0.98\\
493	0.9825\\
494	0.985\\
495	0.9875\\
496	0.99\\
497	0.9925\\
498	0.995\\
499	0.9975\\
500	1\\
501	0.9975\\
502	0.995\\
503	0.9925\\
504	0.99\\
505	0.9875\\
506	0.985\\
507	0.9825\\
508	0.98\\
509	0.9775\\
510	0.975\\
511	0.9725\\
512	0.97\\
513	0.9675\\
514	0.965\\
515	0.9625\\
516	0.96\\
517	0.9575\\
518	0.955\\
519	0.9525\\
520	0.95\\
521	0.9475\\
522	0.945\\
523	0.9425\\
524	0.94\\
525	0.9375\\
526	0.935\\
527	0.9325\\
528	0.93\\
529	0.9275\\
530	0.925\\
531	0.9225\\
532	0.92\\
533	0.9175\\
534	0.915\\
535	0.9125\\
536	0.91\\
537	0.9075\\
538	0.905\\
539	0.9025\\
540	0.9\\
541	0.8975\\
542	0.895\\
543	0.8925\\
544	0.89\\
545	0.8875\\
546	0.885\\
547	0.8825\\
548	0.88\\
549	0.8775\\
550	0.875\\
551	0.8725\\
552	0.87\\
553	0.8675\\
554	0.865\\
555	0.8625\\
556	0.86\\
557	0.8575\\
558	0.855\\
559	0.8525\\
560	0.85\\
561	0.8475\\
562	0.845\\
563	0.8425\\
564	0.84\\
565	0.8375\\
566	0.835\\
567	0.8325\\
568	0.83\\
569	0.8275\\
570	0.825\\
571	0.8225\\
572	0.82\\
573	0.8175\\
574	0.815\\
575	0.8125\\
576	0.81\\
577	0.8075\\
578	0.805\\
579	0.8025\\
580	0.8\\
581	0.7975\\
582	0.795\\
583	0.7925\\
584	0.79\\
585	0.7875\\
586	0.785\\
587	0.7825\\
588	0.78\\
589	0.7775\\
590	0.775\\
591	0.7725\\
592	0.77\\
593	0.7675\\
594	0.765\\
595	0.7625\\
596	0.76\\
597	0.7575\\
598	0.755\\
599	0.7525\\
600	0.75\\
601	0.7475\\
602	0.745\\
603	0.7425\\
604	0.74\\
605	0.7375\\
606	0.735\\
607	0.7325\\
608	0.73\\
609	0.7275\\
610	0.725\\
611	0.7225\\
612	0.72\\
613	0.7175\\
614	0.715\\
615	0.7125\\
616	0.71\\
617	0.7075\\
618	0.705\\
619	0.7025\\
620	0.7\\
621	0.6975\\
622	0.695\\
623	0.6925\\
624	0.69\\
625	0.6875\\
626	0.685\\
627	0.6825\\
628	0.68\\
629	0.6775\\
630	0.675\\
631	0.6725\\
632	0.67\\
633	0.6675\\
634	0.665\\
635	0.6625\\
636	0.66\\
637	0.6575\\
638	0.655\\
639	0.6525\\
640	0.65\\
641	0.6475\\
642	0.645\\
643	0.6425\\
644	0.64\\
645	0.6375\\
646	0.635\\
647	0.6325\\
648	0.63\\
649	0.6275\\
650	0.625\\
651	0.6225\\
652	0.62\\
653	0.6175\\
654	0.615\\
655	0.6125\\
656	0.61\\
657	0.6075\\
658	0.605\\
659	0.6025\\
660	0.6\\
661	0.5975\\
662	0.595\\
663	0.5925\\
664	0.59\\
665	0.5875\\
666	0.585\\
667	0.5825\\
668	0.58\\
669	0.5775\\
670	0.575\\
671	0.5725\\
672	0.57\\
673	0.5675\\
674	0.565\\
675	0.5625\\
676	0.56\\
677	0.5575\\
678	0.555\\
679	0.5525\\
680	0.55\\
681	0.5475\\
682	0.545\\
683	0.5425\\
684	0.54\\
685	0.5375\\
686	0.535\\
687	0.5325\\
688	0.53\\
689	0.5275\\
690	0.525\\
691	0.5225\\
692	0.52\\
693	0.5175\\
694	0.515\\
695	0.5125\\
696	0.51\\
697	0.5075\\
698	0.505\\
699	0.5025\\
700	0.5\\
701	0.4975\\
702	0.495\\
703	0.4925\\
704	0.49\\
705	0.4875\\
706	0.485\\
707	0.4825\\
708	0.48\\
709	0.4775\\
710	0.475\\
711	0.4725\\
712	0.47\\
713	0.4675\\
714	0.465\\
715	0.4625\\
716	0.46\\
717	0.4575\\
718	0.455\\
719	0.4525\\
720	0.45\\
721	0.4475\\
722	0.445\\
723	0.4425\\
724	0.44\\
725	0.4375\\
726	0.435\\
727	0.4325\\
728	0.43\\
729	0.4275\\
730	0.425\\
731	0.4225\\
732	0.42\\
733	0.4175\\
734	0.415\\
735	0.4125\\
736	0.41\\
737	0.4075\\
738	0.405\\
739	0.4025\\
740	0.4\\
741	0.3975\\
742	0.395\\
743	0.3925\\
744	0.39\\
745	0.3875\\
746	0.385\\
747	0.3825\\
748	0.38\\
749	0.3775\\
750	0.375\\
751	0.3725\\
752	0.37\\
753	0.3675\\
754	0.365\\
755	0.3625\\
756	0.36\\
757	0.3575\\
758	0.355\\
759	0.3525\\
760	0.35\\
761	0.3475\\
762	0.345\\
763	0.3425\\
764	0.34\\
765	0.3375\\
766	0.335\\
767	0.3325\\
768	0.33\\
769	0.3275\\
770	0.325\\
771	0.3225\\
772	0.32\\
773	0.3175\\
774	0.315\\
775	0.3125\\
776	0.31\\
777	0.3075\\
778	0.305\\
779	0.3025\\
780	0.3\\
781	0.2975\\
782	0.295\\
783	0.2925\\
784	0.29\\
785	0.2875\\
786	0.285\\
787	0.2825\\
788	0.28\\
789	0.2775\\
790	0.275\\
791	0.2725\\
792	0.27\\
793	0.2675\\
794	0.265\\
795	0.2625\\
796	0.26\\
797	0.2575\\
798	0.255\\
799	0.2525\\
800	0.25\\
801	0.2475\\
802	0.245\\
803	0.2425\\
804	0.24\\
805	0.2375\\
806	0.235\\
807	0.2325\\
808	0.23\\
809	0.2275\\
810	0.225\\
811	0.2225\\
812	0.22\\
813	0.2175\\
814	0.215\\
815	0.2125\\
816	0.21\\
817	0.2075\\
818	0.205\\
819	0.2025\\
820	0.2\\
821	0.1975\\
822	0.195\\
823	0.1925\\
824	0.19\\
825	0.1875\\
826	0.185\\
827	0.1825\\
828	0.18\\
829	0.1775\\
830	0.175\\
831	0.1725\\
832	0.17\\
833	0.1675\\
834	0.165\\
835	0.1625\\
836	0.16\\
837	0.1575\\
838	0.155\\
839	0.1525\\
840	0.15\\
841	0.1475\\
842	0.145\\
843	0.1425\\
844	0.14\\
845	0.1375\\
846	0.135\\
847	0.1325\\
848	0.13\\
849	0.1275\\
850	0.125\\
851	0.1225\\
852	0.12\\
853	0.1175\\
854	0.115\\
855	0.1125\\
856	0.11\\
857	0.1075\\
858	0.105\\
859	0.1025\\
860	0.1\\
861	0.0975\\
862	0.095\\
863	0.0925\\
864	0.09\\
865	0.0875\\
866	0.085\\
867	0.0825\\
868	0.08\\
869	0.0775\\
870	0.075\\
871	0.0725\\
872	0.07\\
873	0.0675\\
874	0.0649999999999999\\
875	0.0625\\
876	0.0600000000000001\\
877	0.0575\\
878	0.055\\
879	0.0525\\
880	0.05\\
881	0.0475\\
882	0.045\\
883	0.0425\\
884	0.04\\
885	0.0375\\
886	0.035\\
887	0.0325\\
888	0.03\\
889	0.0275\\
890	0.025\\
891	0.0225\\
892	0.02\\
893	0.0175\\
894	0.015\\
895	0.0125\\
896	0.01\\
897	0.00749999999999995\\
898	0.005\\
899	0.00249999999999995\\
900	0\\
901	0\\
902	0\\
903	0\\
904	0\\
905	0\\
906	0\\
907	0\\
908	0\\
909	0\\
910	0\\
911	0\\
912	0\\
913	0\\
914	0\\
915	0\\
916	0\\
917	0\\
918	0\\
919	0\\
920	0\\
921	0\\
922	0\\
923	0\\
924	0\\
925	0\\
926	0\\
927	0\\
928	0\\
929	0\\
930	0\\
931	0\\
932	0\\
933	0\\
934	0\\
935	0\\
936	0\\
937	0\\
938	0\\
939	0\\
940	0\\
941	0\\
942	0\\
943	0\\
944	0\\
945	0\\
946	0\\
947	0\\
948	0\\
949	0\\
950	0\\
951	0\\
952	0\\
953	0\\
954	0\\
955	0\\
956	0\\
957	0\\
958	0\\
959	0\\
960	0\\
961	0\\
962	0\\
963	0\\
964	0\\
965	0\\
966	0\\
967	0\\
968	0\\
969	0\\
970	0\\
971	0\\
972	0\\
973	0\\
974	0\\
975	0\\
976	0\\
977	0\\
978	0\\
979	0\\
980	0\\
981	0\\
982	0\\
983	0\\
984	0\\
985	0\\
986	0\\
987	0\\
988	0\\
989	0\\
990	0\\
991	0\\
992	0\\
993	0\\
994	0\\
995	0\\
996	0\\
997	0\\
998	0\\
999	0\\
1000	0\\
};
\addplot [color=black,solid,line width=1.4pt,forget plot]
  table[row sep=crcr]{0	0\\
1	0\\
2	0\\
3	0\\
4	0\\
5	0\\
6	0\\
7	0\\
8	0\\
9	0\\
10	0\\
11	0\\
12	0\\
13	0\\
14	0\\
15	0\\
16	0\\
17	0\\
18	0\\
19	0\\
20	0\\
21	0\\
22	0\\
23	0\\
24	0\\
25	0\\
26	0\\
27	0\\
28	0\\
29	0\\
30	0\\
31	0\\
32	0\\
33	0\\
34	0\\
35	0\\
36	0\\
37	0\\
38	0\\
39	0\\
40	0\\
41	0\\
42	0\\
43	0\\
44	0\\
45	0\\
46	0\\
47	0\\
48	0\\
49	0\\
50	0\\
51	0\\
52	0\\
53	0\\
54	0\\
55	0\\
56	0\\
57	0\\
58	0\\
59	0\\
60	0\\
61	0\\
62	0\\
63	0\\
64	0\\
65	0\\
66	0\\
67	0\\
68	0\\
69	0\\
70	0\\
71	0\\
72	0\\
73	0\\
74	0\\
75	0\\
76	0\\
77	0\\
78	0\\
79	0\\
80	0\\
81	0\\
82	0\\
83	0\\
84	0\\
85	0\\
86	0\\
87	0\\
88	0\\
89	0\\
90	0\\
91	0\\
92	0\\
93	0\\
94	0\\
95	0\\
96	0\\
97	0\\
98	0\\
99	0\\
100	0\\
101	0\\
102	0\\
103	0\\
104	0\\
105	0\\
106	0\\
107	0\\
108	0\\
109	0\\
110	0\\
111	0\\
112	0\\
113	0\\
114	0\\
115	0\\
116	0\\
117	0\\
118	0\\
119	0\\
120	0\\
121	0\\
122	0\\
123	0\\
124	0\\
125	0\\
126	0\\
127	0\\
128	0\\
129	0\\
130	0\\
131	0\\
132	0\\
133	0\\
134	0\\
135	0\\
136	0\\
137	0\\
138	0\\
139	0\\
140	0\\
141	0\\
142	0\\
143	0\\
144	0\\
145	0\\
146	0\\
147	0\\
148	0\\
149	0\\
150	0\\
151	0\\
152	0\\
153	0\\
154	0\\
155	0\\
156	0\\
157	0\\
158	0\\
159	0\\
160	0\\
161	0\\
162	0\\
163	0\\
164	0\\
165	0\\
166	0\\
167	0\\
168	0\\
169	0\\
170	0\\
171	0\\
172	0\\
173	0\\
174	0\\
175	0\\
176	0\\
177	0\\
178	0\\
179	0\\
180	0\\
181	0\\
182	0\\
183	0\\
184	0\\
185	0\\
186	0\\
187	0\\
188	0\\
189	0\\
190	0\\
191	0\\
192	0\\
193	0\\
194	0\\
195	0\\
196	0\\
197	0\\
198	0\\
199	0\\
200	0\\
201	0\\
202	0\\
203	0\\
204	0\\
205	0\\
206	0\\
207	0\\
208	0\\
209	0\\
210	0\\
211	0\\
212	0\\
213	0\\
214	0\\
215	0\\
216	0\\
217	0\\
218	0\\
219	0\\
220	0\\
221	0\\
222	0\\
223	0\\
224	0\\
225	0\\
226	0\\
227	0\\
228	0\\
229	0\\
230	0\\
231	0\\
232	0\\
233	0\\
234	0\\
235	0\\
236	0\\
237	0\\
238	0\\
239	0\\
240	0\\
241	0\\
242	0\\
243	0\\
244	0\\
245	0\\
246	0\\
247	0\\
248	0\\
249	0\\
250	0\\
251	0\\
252	0\\
253	0\\
254	0\\
255	0\\
256	0\\
257	0\\
258	0\\
259	0\\
260	0\\
261	0\\
262	0\\
263	0\\
264	0\\
265	0\\
266	0\\
267	0\\
268	0\\
269	0\\
270	0\\
271	0\\
272	0\\
273	0\\
274	0\\
275	0\\
276	0\\
277	0\\
278	0\\
279	0\\
280	0\\
281	0\\
282	0\\
283	0\\
284	0\\
285	0\\
286	0\\
287	0\\
288	0\\
289	0\\
290	0\\
291	0\\
292	0\\
293	0\\
294	0\\
295	0\\
296	0\\
297	0\\
298	0\\
299	0\\
300	0\\
301	0\\
302	0\\
303	0\\
304	0\\
305	0\\
306	0\\
307	0\\
308	0\\
309	0\\
310	0\\
311	0\\
312	0\\
313	0\\
314	0\\
315	0\\
316	0\\
317	0\\
318	0\\
319	0\\
320	0\\
321	0\\
322	0\\
323	0\\
324	0\\
325	0\\
326	0\\
327	0\\
328	0\\
329	0\\
330	0\\
331	0\\
332	0\\
333	0\\
334	0\\
335	0\\
336	0\\
337	0\\
338	0\\
339	0\\
340	0\\
341	0\\
342	0\\
343	0\\
344	0\\
345	0\\
346	0\\
347	0\\
348	0\\
349	0\\
350	0\\
351	0\\
352	0\\
353	0\\
354	0\\
355	0\\
356	0\\
357	0\\
358	0\\
359	0\\
360	0\\
361	0\\
362	0\\
363	0\\
364	0\\
365	0\\
366	0\\
367	0\\
368	0\\
369	0\\
370	0\\
371	0\\
372	0\\
373	0\\
374	0\\
375	0\\
376	0\\
377	0\\
378	0\\
379	0\\
380	0\\
381	0\\
382	0\\
383	0\\
384	0\\
385	0\\
386	0\\
387	0\\
388	0\\
389	0\\
390	0\\
391	0\\
392	0\\
393	0\\
394	0\\
395	0\\
396	0\\
397	0\\
398	0\\
399	0\\
400	0\\
401	0\\
402	0\\
403	0\\
404	0\\
405	0\\
406	0\\
407	0\\
408	0\\
409	0\\
410	0\\
411	0\\
412	0\\
413	0\\
414	0\\
415	0\\
416	0\\
417	0\\
418	0\\
419	0\\
420	0\\
421	0\\
422	0\\
423	0\\
424	0\\
425	0\\
426	0\\
427	0\\
428	0\\
429	0\\
430	0\\
431	0\\
432	0\\
433	0\\
434	0\\
435	0\\
436	0\\
437	0\\
438	0\\
439	0\\
440	0\\
441	0\\
442	0\\
443	0\\
444	0\\
445	0\\
446	0\\
447	0\\
448	0\\
449	0\\
450	0\\
451	0\\
452	0\\
453	0\\
454	0\\
455	0\\
456	0\\
457	0\\
458	0\\
459	0\\
460	0\\
461	0\\
462	0\\
463	0\\
464	0\\
465	0\\
466	0\\
467	0\\
468	0\\
469	0\\
470	0\\
471	0\\
472	0\\
473	0\\
474	0\\
475	0\\
476	0\\
477	0\\
478	0\\
479	0\\
480	0\\
481	0\\
482	0\\
483	0\\
484	0\\
485	0\\
486	0\\
487	0\\
488	0\\
489	0\\
490	0\\
491	0\\
492	0\\
493	0\\
494	0\\
495	0\\
496	0\\
497	0\\
498	0\\
499	0\\
500	0\\
501	0\\
502	0\\
503	0\\
504	0\\
505	0\\
506	0\\
507	0\\
508	0\\
509	0\\
510	0\\
511	0\\
512	0\\
513	0\\
514	0\\
515	0\\
516	0\\
517	0\\
518	0\\
519	0\\
520	0\\
521	0\\
522	0\\
523	0\\
524	0\\
525	0\\
526	0\\
527	0\\
528	0\\
529	0\\
530	0\\
531	0\\
532	0\\
533	0\\
534	0\\
535	0\\
536	0\\
537	0\\
538	0\\
539	0\\
540	0\\
541	0\\
542	0\\
543	0\\
544	0\\
545	0\\
546	0\\
547	0\\
548	0\\
549	0\\
550	0\\
551	0\\
552	0\\
553	0\\
554	0\\
555	0\\
556	0\\
557	0\\
558	0\\
559	0\\
560	0\\
561	0\\
562	0\\
563	0\\
564	0\\
565	0\\
566	0\\
567	0\\
568	0\\
569	0\\
570	0\\
571	0\\
572	0\\
573	0\\
574	0\\
575	0\\
576	0\\
577	0\\
578	0\\
579	0\\
580	0\\
581	0\\
582	0\\
583	0\\
584	0\\
585	0\\
586	0\\
587	0\\
588	0\\
589	0\\
590	0\\
591	0\\
592	0\\
593	0\\
594	0\\
595	0\\
596	0\\
597	0\\
598	0\\
599	0\\
600	0\\
601	0.0025\\
602	0.005\\
603	0.0075\\
604	0.01\\
605	0.0125\\
606	0.015\\
607	0.0175\\
608	0.02\\
609	0.0225\\
610	0.025\\
611	0.0275\\
612	0.03\\
613	0.0325\\
614	0.035\\
615	0.0375\\
616	0.04\\
617	0.0425\\
618	0.045\\
619	0.0475\\
620	0.05\\
621	0.0525\\
622	0.055\\
623	0.0575\\
624	0.06\\
625	0.0625\\
626	0.065\\
627	0.0675\\
628	0.07\\
629	0.0725\\
630	0.075\\
631	0.0775\\
632	0.08\\
633	0.0825\\
634	0.085\\
635	0.0875\\
636	0.09\\
637	0.0925\\
638	0.095\\
639	0.0975\\
640	0.1\\
641	0.1025\\
642	0.105\\
643	0.1075\\
644	0.11\\
645	0.1125\\
646	0.115\\
647	0.1175\\
648	0.12\\
649	0.1225\\
650	0.125\\
651	0.1275\\
652	0.13\\
653	0.1325\\
654	0.135\\
655	0.1375\\
656	0.14\\
657	0.1425\\
658	0.145\\
659	0.1475\\
660	0.15\\
661	0.1525\\
662	0.155\\
663	0.1575\\
664	0.16\\
665	0.1625\\
666	0.165\\
667	0.1675\\
668	0.17\\
669	0.1725\\
670	0.175\\
671	0.1775\\
672	0.18\\
673	0.1825\\
674	0.185\\
675	0.1875\\
676	0.19\\
677	0.1925\\
678	0.195\\
679	0.1975\\
680	0.2\\
681	0.2025\\
682	0.205\\
683	0.2075\\
684	0.21\\
685	0.2125\\
686	0.215\\
687	0.2175\\
688	0.22\\
689	0.2225\\
690	0.225\\
691	0.2275\\
692	0.23\\
693	0.2325\\
694	0.235\\
695	0.2375\\
696	0.24\\
697	0.2425\\
698	0.245\\
699	0.2475\\
700	0.25\\
701	0.2525\\
702	0.255\\
703	0.2575\\
704	0.26\\
705	0.2625\\
706	0.265\\
707	0.2675\\
708	0.27\\
709	0.2725\\
710	0.275\\
711	0.2775\\
712	0.28\\
713	0.2825\\
714	0.285\\
715	0.2875\\
716	0.29\\
717	0.2925\\
718	0.295\\
719	0.2975\\
720	0.3\\
721	0.3025\\
722	0.305\\
723	0.3075\\
724	0.31\\
725	0.3125\\
726	0.315\\
727	0.3175\\
728	0.32\\
729	0.3225\\
730	0.325\\
731	0.3275\\
732	0.33\\
733	0.3325\\
734	0.335\\
735	0.3375\\
736	0.34\\
737	0.3425\\
738	0.345\\
739	0.3475\\
740	0.35\\
741	0.3525\\
742	0.355\\
743	0.3575\\
744	0.36\\
745	0.3625\\
746	0.365\\
747	0.3675\\
748	0.37\\
749	0.3725\\
750	0.375\\
751	0.3775\\
752	0.38\\
753	0.3825\\
754	0.385\\
755	0.3875\\
756	0.39\\
757	0.3925\\
758	0.395\\
759	0.3975\\
760	0.4\\
761	0.4025\\
762	0.405\\
763	0.4075\\
764	0.41\\
765	0.4125\\
766	0.415\\
767	0.4175\\
768	0.42\\
769	0.4225\\
770	0.425\\
771	0.4275\\
772	0.43\\
773	0.4325\\
774	0.435\\
775	0.4375\\
776	0.44\\
777	0.4425\\
778	0.445\\
779	0.4475\\
780	0.45\\
781	0.4525\\
782	0.455\\
783	0.4575\\
784	0.46\\
785	0.4625\\
786	0.465\\
787	0.4675\\
788	0.47\\
789	0.4725\\
790	0.475\\
791	0.4775\\
792	0.48\\
793	0.4825\\
794	0.485\\
795	0.4875\\
796	0.49\\
797	0.4925\\
798	0.495\\
799	0.4975\\
800	0.5\\
801	0.5025\\
802	0.505\\
803	0.5075\\
804	0.51\\
805	0.5125\\
806	0.515\\
807	0.5175\\
808	0.52\\
809	0.5225\\
810	0.525\\
811	0.5275\\
812	0.53\\
813	0.5325\\
814	0.535\\
815	0.5375\\
816	0.54\\
817	0.5425\\
818	0.545\\
819	0.5475\\
820	0.55\\
821	0.5525\\
822	0.555\\
823	0.5575\\
824	0.56\\
825	0.5625\\
826	0.565\\
827	0.5675\\
828	0.57\\
829	0.5725\\
830	0.575\\
831	0.5775\\
832	0.58\\
833	0.5825\\
834	0.585\\
835	0.5875\\
836	0.59\\
837	0.5925\\
838	0.595\\
839	0.5975\\
840	0.6\\
841	0.6025\\
842	0.605\\
843	0.6075\\
844	0.61\\
845	0.6125\\
846	0.615\\
847	0.6175\\
848	0.62\\
849	0.6225\\
850	0.625\\
851	0.6275\\
852	0.63\\
853	0.6325\\
854	0.635\\
855	0.6375\\
856	0.64\\
857	0.6425\\
858	0.645\\
859	0.6475\\
860	0.65\\
861	0.6525\\
862	0.655\\
863	0.6575\\
864	0.66\\
865	0.6625\\
866	0.665\\
867	0.6675\\
868	0.67\\
869	0.6725\\
870	0.675\\
871	0.6775\\
872	0.68\\
873	0.6825\\
874	0.685\\
875	0.6875\\
876	0.69\\
877	0.6925\\
878	0.695\\
879	0.6975\\
880	0.7\\
881	0.7025\\
882	0.705\\
883	0.7075\\
884	0.71\\
885	0.7125\\
886	0.715\\
887	0.7175\\
888	0.72\\
889	0.7225\\
890	0.725\\
891	0.7275\\
892	0.73\\
893	0.7325\\
894	0.735\\
895	0.7375\\
896	0.74\\
897	0.7425\\
898	0.745\\
899	0.7475\\
900	0.75\\
901	0.7525\\
902	0.755\\
903	0.7575\\
904	0.76\\
905	0.7625\\
906	0.765\\
907	0.7675\\
908	0.77\\
909	0.7725\\
910	0.775\\
911	0.7775\\
912	0.78\\
913	0.7825\\
914	0.785\\
915	0.7875\\
916	0.79\\
917	0.7925\\
918	0.795\\
919	0.7975\\
920	0.8\\
921	0.8025\\
922	0.805\\
923	0.8075\\
924	0.81\\
925	0.8125\\
926	0.815\\
927	0.8175\\
928	0.82\\
929	0.8225\\
930	0.825\\
931	0.8275\\
932	0.83\\
933	0.8325\\
934	0.835\\
935	0.8375\\
936	0.84\\
937	0.8425\\
938	0.845\\
939	0.8475\\
940	0.85\\
941	0.8525\\
942	0.855\\
943	0.8575\\
944	0.86\\
945	0.8625\\
946	0.865\\
947	0.8675\\
948	0.87\\
949	0.8725\\
950	0.875\\
951	0.8775\\
952	0.88\\
953	0.8825\\
954	0.885\\
955	0.8875\\
956	0.89\\
957	0.8925\\
958	0.895\\
959	0.8975\\
960	0.9\\
961	0.9025\\
962	0.905\\
963	0.9075\\
964	0.91\\
965	0.9125\\
966	0.915\\
967	0.9175\\
968	0.92\\
969	0.9225\\
970	0.925\\
971	0.9275\\
972	0.93\\
973	0.9325\\
974	0.935\\
975	0.9375\\
976	0.94\\
977	0.9425\\
978	0.945\\
979	0.9475\\
980	0.95\\
981	0.9525\\
982	0.955\\
983	0.9575\\
984	0.96\\
985	0.9625\\
986	0.965\\
987	0.9675\\
988	0.97\\
989	0.9725\\
990	0.975\\
991	0.9775\\
992	0.98\\
993	0.9825\\
994	0.985\\
995	0.9875\\
996	0.99\\
997	0.9925\\
998	0.995\\
999	0.9975\\
1000	1\\
};
\node[right, inner sep=0mm, text=black]
at (axis cs:1,1.15,0) {L};
\node[right, inner sep=0mm, text=black]
at (axis cs:450,1.15,0) {M};
\node[right, inner sep=0mm, text=black]
at (axis cs:900,1.15,0) {H};
\end{axis}
\end{tikzpicture}%
		\caption{$\chi_3$ - Luz (lux)}
		\label{fig:light-lang-variable}
		\vspace*{2mm}
	\end{subfigure}
	\qquad
	\begin{subfigure}[b]{0.45\textwidth}
		\setlength\figureheight{1.35cm}
		\setlength\figurewidth{4.5cm}
		% This file was created by matlab2tikz v0.4.7 (commit 24ad43488791a40a95315e042d4ef2f4890dac4f) running on MATLAB 8.0.
% Copyright (c) 2008--2014, Nico Schlömer <nico.schloemer@gmail.com>
% All rights reserved.
% Minimal pgfplots version: 1.3
% 
\begin{tikzpicture}

\begin{axis}[%
width=\figurewidth,
height=\figureheight,
clip=false,
scale only axis,
xmin=0,
xmax=100,
ymin=0,
ymax=1
]
\addplot [color=black,solid,line width=1.4pt,forget plot]
  table[row sep=crcr]{0	1\\
1	0.975\\
2	0.95\\
3	0.925\\
4	0.9\\
5	0.875\\
6	0.85\\
7	0.825\\
8	0.8\\
9	0.775\\
10	0.75\\
11	0.725\\
12	0.7\\
13	0.675\\
14	0.65\\
15	0.625\\
16	0.6\\
17	0.575\\
18	0.55\\
19	0.525\\
20	0.5\\
21	0.475\\
22	0.45\\
23	0.425\\
24	0.4\\
25	0.375\\
26	0.35\\
27	0.325\\
28	0.3\\
29	0.275\\
30	0.25\\
31	0.225\\
32	0.2\\
33	0.175\\
34	0.15\\
35	0.125\\
36	0.1\\
37	0.075\\
38	0.05\\
39	0.025\\
40	0\\
41	0\\
42	0\\
43	0\\
44	0\\
45	0\\
46	0\\
47	0\\
48	0\\
49	0\\
50	0\\
51	0\\
52	0\\
53	0\\
54	0\\
55	0\\
56	0\\
57	0\\
58	0\\
59	0\\
60	0\\
61	0\\
62	0\\
63	0\\
64	0\\
65	0\\
66	0\\
67	0\\
68	0\\
69	0\\
70	0\\
71	0\\
72	0\\
73	0\\
74	0\\
75	0\\
76	0\\
77	0\\
78	0\\
79	0\\
80	0\\
81	0\\
82	0\\
83	0\\
84	0\\
85	0\\
86	0\\
87	0\\
88	0\\
89	0\\
90	0\\
91	0\\
92	0\\
93	0\\
94	0\\
95	0\\
96	0\\
97	0\\
98	0\\
99	0\\
100	0\\
};
\addplot [color=black,solid,line width=1.4pt,forget plot]
  table[row sep=crcr]{0	0\\
1	0\\
2	0\\
3	0\\
4	0\\
5	0\\
6	0\\
7	0\\
8	0\\
9	0\\
10	0\\
11	0.025\\
12	0.05\\
13	0.075\\
14	0.1\\
15	0.125\\
16	0.15\\
17	0.175\\
18	0.2\\
19	0.225\\
20	0.25\\
21	0.275\\
22	0.3\\
23	0.325\\
24	0.35\\
25	0.375\\
26	0.4\\
27	0.425\\
28	0.45\\
29	0.475\\
30	0.5\\
31	0.525\\
32	0.55\\
33	0.575\\
34	0.6\\
35	0.625\\
36	0.65\\
37	0.675\\
38	0.7\\
39	0.725\\
40	0.75\\
41	0.775\\
42	0.8\\
43	0.825\\
44	0.85\\
45	0.875\\
46	0.9\\
47	0.925\\
48	0.95\\
49	0.975\\
50	1\\
51	0.975\\
52	0.95\\
53	0.925\\
54	0.9\\
55	0.875\\
56	0.85\\
57	0.825\\
58	0.8\\
59	0.775\\
60	0.75\\
61	0.725\\
62	0.7\\
63	0.675\\
64	0.65\\
65	0.625\\
66	0.6\\
67	0.575\\
68	0.55\\
69	0.525\\
70	0.5\\
71	0.475\\
72	0.45\\
73	0.425\\
74	0.4\\
75	0.375\\
76	0.35\\
77	0.325\\
78	0.3\\
79	0.275\\
80	0.25\\
81	0.225\\
82	0.2\\
83	0.175\\
84	0.15\\
85	0.125\\
86	0.1\\
87	0.075\\
88	0.05\\
89	0.025\\
90	0\\
91	0\\
92	0\\
93	0\\
94	0\\
95	0\\
96	0\\
97	0\\
98	0\\
99	0\\
100	0\\
};
\addplot [color=black,solid,line width=1.4pt,forget plot]
  table[row sep=crcr]{0	0\\
1	0\\
2	0\\
3	0\\
4	0\\
5	0\\
6	0\\
7	0\\
8	0\\
9	0\\
10	0\\
11	0\\
12	0\\
13	0\\
14	0\\
15	0\\
16	0\\
17	0\\
18	0\\
19	0\\
20	0\\
21	0\\
22	0\\
23	0\\
24	0\\
25	0\\
26	0\\
27	0\\
28	0\\
29	0\\
30	0\\
31	0\\
32	0\\
33	0\\
34	0\\
35	0\\
36	0\\
37	0\\
38	0\\
39	0\\
40	0\\
41	0\\
42	0\\
43	0\\
44	0\\
45	0\\
46	0\\
47	0\\
48	0\\
49	0\\
50	0\\
51	0\\
52	0\\
53	0\\
54	0\\
55	0\\
56	0\\
57	0\\
58	0\\
59	0\\
60	0\\
61	0.025\\
62	0.05\\
63	0.075\\
64	0.1\\
65	0.125\\
66	0.15\\
67	0.175\\
68	0.2\\
69	0.225\\
70	0.25\\
71	0.275\\
72	0.3\\
73	0.325\\
74	0.35\\
75	0.375\\
76	0.4\\
77	0.425\\
78	0.45\\
79	0.475\\
80	0.5\\
81	0.525\\
82	0.55\\
83	0.575\\
84	0.6\\
85	0.625\\
86	0.65\\
87	0.675\\
88	0.7\\
89	0.725\\
90	0.75\\
91	0.775\\
92	0.8\\
93	0.825\\
94	0.85\\
95	0.875\\
96	0.9\\
97	0.925\\
98	0.95\\
99	0.975\\
100	1\\
};
\node[right, inner sep=0mm, text=black]
at (axis cs:1,1.15,0) {L};
\node[right, inner sep=0mm, text=black]
at (axis cs:45,1.15,0) {M};
\node[right, inner sep=0mm, text=black]
at (axis cs:90,1.15,0) {H};
\end{axis}
\end{tikzpicture}%
		\caption{$\chi_4$ - Humedad (ppm)}
		\label{fig:humidity-lang-variable}
		\vspace*{2mm}
	\end{subfigure}

	\begin{subfigure}[b]{0.45\textwidth}
		\setlength\figureheight{1.35cm}
		\setlength\figurewidth{4.5cm}
		% This file was created by matlab2tikz v0.4.7 (commit 24ad43488791a40a95315e042d4ef2f4890dac4f) running on MATLAB 8.0.
% Copyright (c) 2008--2014, Nico Schlömer <nico.schloemer@gmail.com>
% All rights reserved.
% Minimal pgfplots version: 1.3
% 
\begin{tikzpicture}

\begin{axis}[%
width=\figurewidth,
height=\figureheight,
clip=false,
scale only axis,
xmin=0,
xmax=80,
ymin=0,
ymax=1
]
\addplot [color=black,solid,line width=1.4pt,forget plot]
  table[row sep=crcr]{0	1\\
1	0.966666666666667\\
2	0.933333333333333\\
3	0.9\\
4	0.866666666666667\\
5	0.833333333333333\\
6	0.8\\
7	0.766666666666667\\
8	0.733333333333333\\
9	0.7\\
10	0.666666666666667\\
11	0.633333333333333\\
12	0.6\\
13	0.566666666666667\\
14	0.533333333333333\\
15	0.5\\
16	0.466666666666667\\
17	0.433333333333333\\
18	0.4\\
19	0.366666666666667\\
20	0.333333333333333\\
21	0.3\\
22	0.266666666666667\\
23	0.233333333333333\\
24	0.2\\
25	0.166666666666667\\
26	0.133333333333333\\
27	0.1\\
28	0.0666666666666667\\
29	0.0333333333333333\\
30	0\\
31	0\\
32	0\\
33	0\\
34	0\\
35	0\\
36	0\\
37	0\\
38	0\\
39	0\\
40	0\\
41	0\\
42	0\\
43	0\\
44	0\\
45	0\\
46	0\\
47	0\\
48	0\\
49	0\\
50	0\\
51	0\\
52	0\\
53	0\\
54	0\\
55	0\\
56	0\\
57	0\\
58	0\\
59	0\\
60	0\\
61	0\\
62	0\\
63	0\\
64	0\\
65	0\\
66	0\\
67	0\\
68	0\\
69	0\\
70	0\\
71	0\\
72	0\\
73	0\\
74	0\\
75	0\\
76	0\\
77	0\\
78	0\\
79	0\\
80	0\\
};
\addplot [color=black,solid,line width=1.4pt,forget plot]
  table[row sep=crcr]{0	0\\
1	0\\
2	0\\
3	0\\
4	0\\
5	0\\
6	0\\
7	0\\
8	0\\
9	0\\
10	0\\
11	0.0333333333333333\\
12	0.0666666666666667\\
13	0.1\\
14	0.133333333333333\\
15	0.166666666666667\\
16	0.2\\
17	0.233333333333333\\
18	0.266666666666667\\
19	0.3\\
20	0.333333333333333\\
21	0.366666666666667\\
22	0.4\\
23	0.433333333333333\\
24	0.466666666666667\\
25	0.5\\
26	0.533333333333333\\
27	0.566666666666667\\
28	0.6\\
29	0.633333333333333\\
30	0.666666666666667\\
31	0.7\\
32	0.733333333333333\\
33	0.766666666666667\\
34	0.8\\
35	0.833333333333333\\
36	0.866666666666667\\
37	0.9\\
38	0.933333333333333\\
39	0.966666666666667\\
40	1\\
41	0.966666666666667\\
42	0.933333333333333\\
43	0.9\\
44	0.866666666666667\\
45	0.833333333333333\\
46	0.8\\
47	0.766666666666667\\
48	0.733333333333333\\
49	0.7\\
50	0.666666666666667\\
51	0.633333333333333\\
52	0.6\\
53	0.566666666666667\\
54	0.533333333333333\\
55	0.5\\
56	0.466666666666667\\
57	0.433333333333333\\
58	0.4\\
59	0.366666666666667\\
60	0.333333333333333\\
61	0.3\\
62	0.266666666666667\\
63	0.233333333333333\\
64	0.2\\
65	0.166666666666667\\
66	0.133333333333333\\
67	0.1\\
68	0.0666666666666667\\
69	0.0333333333333333\\
70	0\\
71	0\\
72	0\\
73	0\\
74	0\\
75	0\\
76	0\\
77	0\\
78	0\\
79	0\\
80	0\\
};
\addplot [color=black,solid,line width=1.4pt,forget plot]
  table[row sep=crcr]{0	0\\
1	0\\
2	0\\
3	0\\
4	0\\
5	0\\
6	0\\
7	0\\
8	0\\
9	0\\
10	0\\
11	0\\
12	0\\
13	0\\
14	0\\
15	0\\
16	0\\
17	0\\
18	0\\
19	0\\
20	0\\
21	0\\
22	0\\
23	0\\
24	0\\
25	0\\
26	0\\
27	0\\
28	0\\
29	0\\
30	0\\
31	0\\
32	0\\
33	0\\
34	0\\
35	0\\
36	0\\
37	0\\
38	0\\
39	0\\
40	0\\
41	0\\
42	0\\
43	0\\
44	0\\
45	0\\
46	0\\
47	0\\
48	0\\
49	0\\
50	0\\
51	0.0333333333333333\\
52	0.0666666666666667\\
53	0.1\\
54	0.133333333333333\\
55	0.166666666666667\\
56	0.2\\
57	0.233333333333333\\
58	0.266666666666667\\
59	0.3\\
60	0.333333333333333\\
61	0.366666666666667\\
62	0.4\\
63	0.433333333333333\\
64	0.466666666666667\\
65	0.5\\
66	0.533333333333333\\
67	0.566666666666667\\
68	0.6\\
69	0.633333333333333\\
70	0.666666666666667\\
71	0.7\\
72	0.733333333333333\\
73	0.766666666666667\\
74	0.8\\
75	0.833333333333333\\
76	0.866666666666667\\
77	0.9\\
78	0.933333333333333\\
79	0.966666666666667\\
80	1\\
};
\node[right, inner sep=0mm, text=black]
at (axis cs:1,1.15,0) {L};
\node[right, inner sep=0mm, text=black]
at (axis cs:38,1.15,0) {M};
\node[right, inner sep=0mm, text=black]
at (axis cs:75,1.15,0) {H};
\end{axis}
\end{tikzpicture}%
		\caption{$\chi_5$ - Distancia (m)}
		\label{fig:distance-lang-variable}
		\vspace*{2mm}
	\end{subfigure}
	\qquad
	\begin{subfigure}[b]{0.45\textwidth}
		\setlength\figureheight{1.35cm}
		\setlength\figurewidth{4.5cm}
		% This file was created by matlab2tikz v0.4.7 (commit 24ad43488791a40a95315e042d4ef2f4890dac4f) running on MATLAB 8.0.
% Copyright (c) 2008--2014, Nico Schlömer <nico.schloemer@gmail.com>
% All rights reserved.
% Minimal pgfplots version: 1.3
% 
\begin{tikzpicture}

\begin{axis}[%
width=\figurewidth,
height=\figureheight,
clip=false,
scale only axis,
xmin=0,
xmax=100,
ymin=0,
ymax=1
]
\addplot [color=black,solid,line width=1.4pt,forget plot]
  table[row sep=crcr]{0	1\\
1	0.96\\
2	0.92\\
3	0.88\\
4	0.84\\
5	0.8\\
6	0.76\\
7	0.72\\
8	0.68\\
9	0.64\\
10	0.6\\
11	0.56\\
12	0.52\\
13	0.48\\
14	0.44\\
15	0.4\\
16	0.36\\
17	0.32\\
18	0.28\\
19	0.24\\
20	0.2\\
21	0.16\\
22	0.12\\
23	0.08\\
24	0.04\\
25	0\\
26	0\\
27	0\\
28	0\\
29	0\\
30	0\\
31	0\\
32	0\\
33	0\\
34	0\\
35	0\\
36	0\\
37	0\\
38	0\\
39	0\\
40	0\\
41	0\\
42	0\\
43	0\\
44	0\\
45	0\\
46	0\\
47	0\\
48	0\\
49	0\\
50	0\\
51	0\\
52	0\\
53	0\\
54	0\\
55	0\\
56	0\\
57	0\\
58	0\\
59	0\\
60	0\\
61	0\\
62	0\\
63	0\\
64	0\\
65	0\\
66	0\\
67	0\\
68	0\\
69	0\\
70	0\\
71	0\\
72	0\\
73	0\\
74	0\\
75	0\\
76	0\\
77	0\\
78	0\\
79	0\\
80	0\\
81	0\\
82	0\\
83	0\\
84	0\\
85	0\\
86	0\\
87	0\\
88	0\\
89	0\\
90	0\\
91	0\\
92	0\\
93	0\\
94	0\\
95	0\\
96	0\\
97	0\\
98	0\\
99	0\\
100	0\\
};
\addplot [color=black,solid,line width=1.4pt,forget plot]
  table[row sep=crcr]{0	0\\
1	0.04\\
2	0.08\\
3	0.12\\
4	0.16\\
5	0.2\\
6	0.24\\
7	0.28\\
8	0.32\\
9	0.36\\
10	0.4\\
11	0.44\\
12	0.48\\
13	0.52\\
14	0.56\\
15	0.6\\
16	0.64\\
17	0.68\\
18	0.72\\
19	0.76\\
20	0.8\\
21	0.84\\
22	0.88\\
23	0.92\\
24	0.96\\
25	1\\
26	0.96\\
27	0.92\\
28	0.88\\
29	0.84\\
30	0.8\\
31	0.76\\
32	0.72\\
33	0.68\\
34	0.64\\
35	0.6\\
36	0.56\\
37	0.52\\
38	0.48\\
39	0.44\\
40	0.4\\
41	0.36\\
42	0.32\\
43	0.28\\
44	0.24\\
45	0.2\\
46	0.16\\
47	0.12\\
48	0.08\\
49	0.04\\
50	0\\
51	0\\
52	0\\
53	0\\
54	0\\
55	0\\
56	0\\
57	0\\
58	0\\
59	0\\
60	0\\
61	0\\
62	0\\
63	0\\
64	0\\
65	0\\
66	0\\
67	0\\
68	0\\
69	0\\
70	0\\
71	0\\
72	0\\
73	0\\
74	0\\
75	0\\
76	0\\
77	0\\
78	0\\
79	0\\
80	0\\
81	0\\
82	0\\
83	0\\
84	0\\
85	0\\
86	0\\
87	0\\
88	0\\
89	0\\
90	0\\
91	0\\
92	0\\
93	0\\
94	0\\
95	0\\
96	0\\
97	0\\
98	0\\
99	0\\
100	0\\
};
\addplot [color=black,solid,line width=1.4pt,forget plot]
  table[row sep=crcr]{0	0\\
1	0\\
2	0\\
3	0\\
4	0\\
5	0\\
6	0\\
7	0\\
8	0\\
9	0\\
10	0\\
11	0\\
12	0\\
13	0\\
14	0\\
15	0\\
16	0\\
17	0\\
18	0\\
19	0\\
20	0\\
21	0\\
22	0\\
23	0\\
24	0\\
25	0\\
26	0.04\\
27	0.08\\
28	0.12\\
29	0.16\\
30	0.2\\
31	0.24\\
32	0.28\\
33	0.32\\
34	0.36\\
35	0.4\\
36	0.44\\
37	0.48\\
38	0.52\\
39	0.56\\
40	0.6\\
41	0.64\\
42	0.68\\
43	0.72\\
44	0.76\\
45	0.8\\
46	0.84\\
47	0.88\\
48	0.92\\
49	0.96\\
50	1\\
51	0.96\\
52	0.92\\
53	0.88\\
54	0.84\\
55	0.8\\
56	0.76\\
57	0.72\\
58	0.68\\
59	0.64\\
60	0.6\\
61	0.56\\
62	0.52\\
63	0.48\\
64	0.44\\
65	0.4\\
66	0.36\\
67	0.32\\
68	0.28\\
69	0.24\\
70	0.2\\
71	0.16\\
72	0.12\\
73	0.08\\
74	0.04\\
75	0\\
76	0\\
77	0\\
78	0\\
79	0\\
80	0\\
81	0\\
82	0\\
83	0\\
84	0\\
85	0\\
86	0\\
87	0\\
88	0\\
89	0\\
90	0\\
91	0\\
92	0\\
93	0\\
94	0\\
95	0\\
96	0\\
97	0\\
98	0\\
99	0\\
100	0\\
};
\addplot [color=black,solid,line width=1.4pt,forget plot]
  table[row sep=crcr]{0	0\\
1	0\\
2	0\\
3	0\\
4	0\\
5	0\\
6	0\\
7	0\\
8	0\\
9	0\\
10	0\\
11	0\\
12	0\\
13	0\\
14	0\\
15	0\\
16	0\\
17	0\\
18	0\\
19	0\\
20	0\\
21	0\\
22	0\\
23	0\\
24	0\\
25	0\\
26	0\\
27	0\\
28	0\\
29	0\\
30	0\\
31	0\\
32	0\\
33	0\\
34	0\\
35	0\\
36	0\\
37	0\\
38	0\\
39	0\\
40	0\\
41	0\\
42	0\\
43	0\\
44	0\\
45	0\\
46	0\\
47	0\\
48	0\\
49	0\\
50	0\\
51	0.04\\
52	0.08\\
53	0.12\\
54	0.16\\
55	0.2\\
56	0.24\\
57	0.28\\
58	0.32\\
59	0.36\\
60	0.4\\
61	0.44\\
62	0.48\\
63	0.52\\
64	0.56\\
65	0.6\\
66	0.64\\
67	0.68\\
68	0.72\\
69	0.76\\
70	0.8\\
71	0.84\\
72	0.88\\
73	0.92\\
74	0.96\\
75	1\\
76	0.96\\
77	0.92\\
78	0.88\\
79	0.84\\
80	0.8\\
81	0.76\\
82	0.72\\
83	0.68\\
84	0.64\\
85	0.6\\
86	0.56\\
87	0.52\\
88	0.48\\
89	0.44\\
90	0.4\\
91	0.36\\
92	0.32\\
93	0.28\\
94	0.24\\
95	0.2\\
96	0.16\\
97	0.12\\
98	0.08\\
99	0.04\\
100	0\\
};
\addplot [color=black,solid,line width=1.4pt,forget plot]
  table[row sep=crcr]{0	0\\
1	0\\
2	0\\
3	0\\
4	0\\
5	0\\
6	0\\
7	0\\
8	0\\
9	0\\
10	0\\
11	0\\
12	0\\
13	0\\
14	0\\
15	0\\
16	0\\
17	0\\
18	0\\
19	0\\
20	0\\
21	0\\
22	0\\
23	0\\
24	0\\
25	0\\
26	0\\
27	0\\
28	0\\
29	0\\
30	0\\
31	0\\
32	0\\
33	0\\
34	0\\
35	0\\
36	0\\
37	0\\
38	0\\
39	0\\
40	0\\
41	0\\
42	0\\
43	0\\
44	0\\
45	0\\
46	0\\
47	0\\
48	0\\
49	0\\
50	0\\
51	0\\
52	0\\
53	0\\
54	0\\
55	0\\
56	0\\
57	0\\
58	0\\
59	0\\
60	0\\
61	0\\
62	0\\
63	0\\
64	0\\
65	0\\
66	0\\
67	0\\
68	0\\
69	0\\
70	0\\
71	0\\
72	0\\
73	0\\
74	0\\
75	0\\
76	0.04\\
77	0.08\\
78	0.12\\
79	0.16\\
80	0.2\\
81	0.24\\
82	0.28\\
83	0.32\\
84	0.36\\
85	0.4\\
86	0.44\\
87	0.48\\
88	0.52\\
89	0.56\\
90	0.6\\
91	0.64\\
92	0.68\\
93	0.72\\
94	0.76\\
95	0.8\\
96	0.84\\
97	0.88\\
98	0.92\\
99	0.96\\
100	1\\
};
\node[right, inner sep=0mm, text=black]
at (axis cs:1,1.15,0) {VL};
\node[right, inner sep=0mm, text=black]
at (axis cs:22,1.15,0) {L};
\node[right, inner sep=0mm, text=black]
at (axis cs:45,1.15,0) {M};
\node[right, inner sep=0mm, text=black]
at (axis cs:72,1.15,0) {H};
\node[right, inner sep=0mm, text=black]
at (axis cs:92,1.15,0) {VH};
\end{axis}
\end{tikzpicture}%
		\caption{$y$ - Riesgo de incendio (\%)}
		\label{fig:threat-lang-variable}
		\vspace*{2mm}
	\end{subfigure}
\end{figure}
\end{frame}

\begin{frame}{Riesgo de incendios forestales: resultados}
\begin{itemize}
\item Riesgos de incendio obtenidos al aplicar el método de Mamdani y el método de interpolación basado en índices de solapamiento (con $M = \text{Media aritmética}$, $T = T_{min}$ y $O = O_Z$) a las entradas:
	\begin{enumerate}
		\item Temperatura: 30ºC.
		\item Humo: 20ppm.
		\item Luz: 500lux.
		\item Humedad: 50ppm.
		\item Distancia: 40m.
	\end{enumerate}
\end{itemize}
\vspace{0.2cm}
\begin{figure}[H]
	\centering
	\begin{subfigure}[b]{0.45\textwidth}
		\caption{Mamdani}
		\setlength\figureheight{2cm}
		\setlength\figurewidth{4.25cm}
		% This file was created by matlab2tikz v0.4.7 running on MATLAB 8.0.
% Copyright (c) 2008--2014, Nico Schlömer <nico.schloemer@gmail.com>
% All rights reserved.
% Minimal pgfplots version: 1.3
% 
%
% defining custom colors
\definecolor{mycolor1}{rgb}{0.00000,1.00000,1.00000}%
\definecolor{mycolor2}{rgb}{1.00000,0.00000,1.00000}%
%
\begin{tikzpicture}

\begin{axis}[%
width=\figurewidth,
height=\figureheight,
scale only axis,
xmin=0,
xmax=100,
xlabel={Riesgo (\%)},
ymin=0,
ymax=0.4,
axis x line*=bottom,
axis y line*=left,
legend style={at={(0.5,1.25)},anchor=south,legend columns=5,draw=black,fill=white,legend cell align=left, font=\tiny,style={column sep=0.05cm}}
]
\addplot [color=black,solid,line width=1.2pt,forget plot]
  table[row sep=crcr]{0	0\\
1	0.04\\
2	0.08\\
3	0.12\\
4	0.16\\
5	0.2\\
6	0.24\\
7	0.28\\
8	0.32\\
9	0.36\\
10	0.4\\
11	0.4\\
12	0.4\\
13	0.4\\
14	0.4\\
15	0.4\\
16	0.4\\
17	0.4\\
18	0.4\\
19	0.4\\
20	0.4\\
21	0.4\\
22	0.4\\
23	0.4\\
24	0.4\\
25	0.4\\
26	0.4\\
27	0.4\\
28	0.4\\
29	0.4\\
30	0.4\\
31	0.4\\
32	0.4\\
33	0.4\\
34	0.4\\
35	0.4\\
36	0.4\\
37	0.4\\
38	0.4\\
39	0.4\\
40	0.4\\
41	0.36\\
42	0.32\\
43	0.28\\
44	0.25\\
45	0.25\\
46	0.25\\
47	0.25\\
48	0.25\\
49	0.25\\
50	0.25\\
51	0.25\\
52	0.25\\
53	0.25\\
54	0.25\\
55	0.25\\
56	0.25\\
57	0.25\\
58	0.25\\
59	0.25\\
60	0.25\\
61	0.25\\
62	0.25\\
63	0.25\\
64	0.25\\
65	0.25\\
66	0.25\\
67	0.25\\
68	0.25\\
69	0.24\\
70	0.2\\
71	0.16\\
72	0.12\\
73	0.08\\
74	0.04\\
75	0\\
76	0\\
77	0\\
78	0\\
79	0\\
80	0\\
81	0\\
82	0\\
83	0\\
84	0\\
85	0\\
86	0\\
87	0\\
88	0\\
89	0\\
90	0\\
91	0\\
92	0\\
93	0\\
94	0\\
95	0\\
96	0\\
97	0\\
98	0\\
99	0\\
100	0\\
};
\addplot [color=blue,line width=1.2pt,mark size=4.5pt,only marks,mark=asterisk,mark options={solid}]
  table[row sep=crcr]{35	0.4\\
};
\addlegendentry{ctr:35};

\addplot [color=green,line width=1.2pt,mark size=4.5pt,only marks,mark=+,mark options={solid}]
  table[row sep=crcr]{33	0.4\\
};
\addlegendentry{bis:33};

\addplot [color=red,line width=1.2pt,mark size=3.2pt,only marks,mark=square,mark options={solid}]
  table[row sep=crcr]{25	0.4\\
};
\addlegendentry{mom:25};

\addplot [color=mycolor1,line width=1.2pt,mark size=3.0pt,only marks,mark=triangle,mark options={solid,rotate=180}]
  table[row sep=crcr]{10	0.4\\
};
\addlegendentry{som:10};

\addplot [color=mycolor2,line width=1.2pt,mark size=3.0pt,only marks,mark=triangle,mark options={solid}]
  table[row sep=crcr]{40	0.4\\
};
\addlegendentry{lom:40};

\end{axis}
\end{tikzpicture}%
	\end{subfigure}
	\qquad
	\begin{subfigure}[b]{0.45\textwidth}
		\caption{Método de interpolación}
		\setlength\figureheight{2cm}
		\setlength\figurewidth{4.25cm}
		% This file was created by matlab2tikz v0.4.7 running on MATLAB 8.0.
% Copyright (c) 2008--2014, Nico Schlömer <nico.schloemer@gmail.com>
% All rights reserved.
% Minimal pgfplots version: 1.3
% 
%
% defining custom colors
\definecolor{mycolor1}{rgb}{0.00000,1.00000,1.00000}%
\definecolor{mycolor2}{rgb}{1.00000,0.00000,1.00000}%
%
\begin{tikzpicture}

\begin{axis}[%
width=\figurewidth,
height=\figureheight,
scale only axis,
xmin=0,
xmax=100,
xlabel={Riesgo (\%)},
ymin=0,
ymax=0.006,
axis x line*=bottom,
axis y line*=left,
legend style={at={(0.5,1.25)},anchor=south,legend columns=5,draw=black,fill=white,legend cell align=left, font=\tiny,style={column sep=0.05cm}}
]
\addplot [color=black,solid,line width=1.2pt,forget plot]
  table[row sep=crcr]{0	0\\
1	0.000493827160493827\\
2	0.000987654320987654\\
3	0.00148148148148148\\
4	0.00197530864197531\\
5	0.00246913580246914\\
6	0.00296296296296296\\
7	0.00333333333333333\\
8	0.00366255144032922\\
9	0.0039917695473251\\
10	0.00432098765432099\\
11	0.00432098765432099\\
12	0.00432098765432099\\
13	0.00432098765432099\\
14	0.00432098765432099\\
15	0.00432098765432099\\
16	0.00432098765432099\\
17	0.00432098765432099\\
18	0.00432098765432099\\
19	0.00432098765432099\\
20	0.00432098765432099\\
21	0.00432098765432099\\
22	0.00432098765432099\\
23	0.00432098765432099\\
24	0.00432098765432099\\
25	0.00432098765432099\\
26	0.00448559670781893\\
27	0.00465020576131687\\
28	0.00481481481481481\\
29	0.00497942386831276\\
30	0.0051440329218107\\
31	0.00530864197530864\\
32	0.00534979423868313\\
33	0.00534979423868313\\
34	0.00534979423868313\\
35	0.00534979423868313\\
36	0.00534979423868313\\
37	0.00534979423868313\\
38	0.00534979423868313\\
39	0.00534979423868313\\
40	0.00534979423868313\\
41	0.00502057613168724\\
42	0.00469135802469136\\
43	0.00436213991769547\\
44	0.0039917695473251\\
45	0.00349794238683128\\
46	0.00300411522633745\\
47	0.00251028806584362\\
48	0.00201646090534979\\
49	0.00152263374485597\\
50	0.00102880658436214\\
51	0.00102880658436214\\
52	0.00102880658436214\\
53	0.00102880658436214\\
54	0.00102880658436214\\
55	0.00102880658436214\\
56	0.00102880658436214\\
57	0.00102880658436214\\
58	0.00102880658436214\\
59	0.00102880658436214\\
60	0.00102880658436214\\
61	0.00102880658436214\\
62	0.00102880658436214\\
63	0.00102880658436214\\
64	0.00102880658436214\\
65	0.00102880658436214\\
66	0.00102880658436214\\
67	0.00102880658436214\\
68	0.00102880658436214\\
69	0.000987654320987654\\
70	0.000823045267489712\\
71	0.00065843621399177\\
72	0.000493827160493827\\
73	0.000329218106995885\\
74	0.000164609053497943\\
75	0\\
76	0\\
77	0\\
78	0\\
79	0\\
80	0\\
81	0\\
82	0\\
83	0\\
84	0\\
85	0\\
86	0\\
87	0\\
88	0\\
89	0\\
90	0\\
91	0\\
92	0\\
93	0\\
94	0\\
95	0\\
96	0\\
97	0\\
98	0\\
99	0\\
100	0\\
};
\addplot [color=blue,line width=1.2pt,mark size=4.5pt,only marks,mark=asterisk,mark options={solid}]
  table[row sep=crcr]{30	0.0051440329218107\\
};
\addlegendentry{ctr:30};

\addplot [color=green,line width=1.2pt,mark size=4.5pt,only marks,mark=+,mark options={solid}]
  table[row sep=crcr]{30	0.0051440329218107\\
};
\addlegendentry{bis:30};

\addplot [color=red,line width=1.2pt,mark size=3.2pt,only marks,mark=square,mark options={solid}]
  table[row sep=crcr]{36	0.00534979423868313\\
};
\addlegendentry{mom:36};

\addplot [color=mycolor1,line width=1.2pt,mark size=3.0pt,only marks,mark=triangle,mark options={solid,rotate=180}]
  table[row sep=crcr]{32	0.00534979423868313\\
};
\addlegendentry{som:32};

\addplot [color=mycolor2,line width=1.2pt,mark size=3.0pt,only marks,mark=triangle,mark options={solid}]
  table[row sep=crcr]{40	0.00534979423868313\\
};
\addlegendentry{lom:40};

\end{axis}
\end{tikzpicture}%
	\end{subfigure}
\end{figure}
\end{frame}

\begin{frame}{Riesgo de incendios forestales: resultados (cont.)}
Riesgos de incendio para diferentes entradas: \\
\vspace{0.3cm}
\textbf{Método de Mamdani}
\begin{center}
	\tiny
	\begin{tabular}{| c | c | c | c | c | c | c | c  | c  | c |}
\hline
 \multirow{2}{*}{\textbf{Temp.}} & \multirow{2}{*}{\textbf{Humo}} & \multirow{2}{*}{\textbf{Luz}}& \multirow{2}{*}{\textbf{Hum.}} & \multirow{2}{*}{\textbf{Dist.}} &  \multicolumn{5}{|c|}{\textbf{Riesgo (\%)}} \\ 
\cline{6-10}
 & & & & & \textbf{centroide  (\textasteriskcentered)} & \textbf{bisector (+)} & \textbf{som ($\triangledown$)} & \textbf{mom ($\square$)} & \textbf{lom ($\vartriangle$)}  \\ 
\hline
25 & 0 & 200 & 20 & 70 & 20 & 18 & 0 & 6 & 12 \\ 
\hline
30 & 20 & 500 & 50 & 40 & 35 & 33 & 10 & 25 & 40 \\
\hline
40 & 50 & 500 & 30 & 40 & 43 & 45 & 38 & 50 & 62 \\
\hline
80 & 80 & 700 & 20 & 30 & 67 & 69 & 63 & 75 & 87 \\
\hline
100 & 90 & 900 & 10 & 20 & 80 & 82 & 84 & 92 & 100 \\
\hline
120 & 100 & 1000 & 10 & 10 & 91 & 92 & 92 & 96 & 100 \\
\hline
\end{tabular}
\end{center}
\vspace{0.5cm}
\textbf{Método de interpolación basado en índices de solapamiento} \\
{\footnotesize   ($M = \text{Media aritmética}$, $T = T_{min}$ y $O = O_Z$)}
\begin{center}
	\tiny
    \begin{tabular}{| c | c | c | c | c | c | c | c  | c  | c |}
    \hline
     \multirow{2}{*}{\textbf{Temp.}} & \multirow{2}{*}{\textbf{Humo}} & \multirow{2}{*}{\textbf{Luz}}& \multirow{2}{*}{\textbf{Hum.}} & \multirow{2}{*}{\textbf{Dist.}} &  \multicolumn{5}{|c|}{\textbf{Riesgo (\%)}} \\ 
    \cline{6-10}
    & & & & & \textbf{centroide  (\textasteriskcentered)} & \textbf{bisector (+)} & \textbf{som ($\triangledown$)} & \textbf{mom ($\square$)} & \textbf{lom ($\vartriangle$)}  \\ 
    \hline
    25 & 0 & 200 & 20 & 70 & 18 & 15 & 8 & 10 & 12  \\ 
    \hline
    30 & 20 & 500 & 50 & 40 & 30 & 30 & 32 & 36 & 40 \\
    \hline
    40 & 50 & 500 & 30 & 40 & 41 & 41 & 38 & 42 & 45 \\
    \hline
    80 & 80 & 700 & 20 & 30 & 69 & 69 & 63 & 66 & 68 \\
    \hline
    100 & 90 & 900 & 10 & 20 & 85 & 87 & 84 & 90 & 95 \\
    \hline
    120 & 100 & 1000 & 10 & 10 & 91 & 92 & 92 & 96 & 100 \\
    \hline
	\end{tabular}
\end{center}
\end{frame}

\begin{frame}
\centering
\footnotesize
$Temp. = 30\text{ºC}$, $Humo = 20 \text{ppm}$, $Luz = 500 \text{lux}$, $Humedad = 50 \text{ppm}$ y $Dist. = 40 \text{m}$
\begin{center}
	\tiny
	\begin{tabular}{| c | c | c | c | c | c | c |}
\hline
 \multirow{2}{*}{\textbf{T-norma}} & \multirow{2}{*}{\textbf{Índice de solapamiento}} &  \multicolumn{5}{|c|}{\textbf{Riesgo (\%)}} \\ \cline{3-7}
& & \textbf{cent.  (\textasteriskcentered)} & \textbf{bis. (+)} & \textbf{som ($\triangledown$)} & \textbf{mom ($\square$)} & \textbf{lom ($\vartriangle$)}  \\ \hline
\multirow{5}{*}{$T_{prod}$}&$O_{\pi}$ &29 &29 &26 &38 &49 \\ \cline{2-7}
&$O_{avgmin}$ &29 &29 &26 &38 &49 \\ \cline{2-7}
&$O_{Z}$ &29 &29 &28 &36 &44 \\ \cline{2-7}
&$O_{\sqrt{\text{ }}}$ &30 &30 &26 &38 &49 \\ \cline{2-7}
&$O_{sin}$ &31 &31 &26 &38 &49 \\ \hhline{|=|=|=|=|=|=|=|}
\multirow{5}{*}{$T_{min}$}&$O_{\pi}$ &31 &31 &26 &38 &49 \\ \cline{2-7}
&$O_{avgmin}$ &31 &31 &26 &38 &49 \\ \cline{2-7}
&$O_{Z}$ &30 &30 &32 &36 &40 \\ \cline{2-7}
&$O_{\sqrt{\text{ }}}$ &31 &31 &26 &38 &49 \\ \cline{2-7}
&$O_{sin}$ &31 &31 &26 &38 &49 \\ \hhline{|=|=|=|=|=|=|=|}
\multirow{5}{*}{$T_{geo}$}&$O_{\pi}$ &31 &31 &26 &38 &49 \\ \cline{2-7}
&$O_{avgmin}$ &31 &31 &26 &38 &49 \\ \cline{2-7}
&$O_{Z}$ &31 &30 &32 &32 &32 \\ \cline{2-7}
&$O_{\sqrt{\text{ }}}$ &31 &31 &26 &38 &49 \\ \cline{2-7}
&$O_{sin}$ &31 &31 &26 &38 &49 \\ \hhline{|=|=|=|=|=|=|=|}
\multirow{5}{*}{$T_{harm}$}&$O_{\pi}$ &31 &31 &26 &38 &49 \\ \cline{2-7}
&$O_{avgmin}$ &31 &31 &26 &38 &49 \\ \cline{2-7}
&$O_{Z}$ &31 &30 &33 &33 &33 \\ \cline{2-7}
&$O_{\sqrt{\text{ }}}$ &31 &31 &26 &38 &49 \\ \cline{2-7}
&$O_{sin}$ &31 &31 &26 &38 &49 \\ \hhline{|=|=|=|=|=|=|=|}
\multirow{5}{*}{$T_{sin}$}&$O_{\pi}$ &31 &31 &26 &38 &49 \\ \cline{2-7}
&$O_{avgmin}$ &31 &31 &26 &38 &49 \\ \cline{2-7}
&$O_{Z}$ &31 &31 &28 &37 &46 \\ \cline{2-7}
&$O_{\sqrt{\text{ }}}$ &31 &31 &26 &38 &49 \\ \cline{2-7}
&$O_{sin}$ &31 &31 &26 &38 &49 \\ \hhline{|=|=|=|=|=|=|=|}
\multirow{5}{*}{$T_{einstein}$}&$O_{\pi}$ &29 &29 &26 &38 &49 \\ \cline{2-7}
&$O_{avgmin}$ &29 &29 &26 &38 &49 \\ \cline{2-7}
&$O_{Z}$ &29 &29 &28 &36 &44 \\ \cline{2-7}
&$O_{\sqrt{\text{ }}}$ &30 &30 &26 &38 &49 \\ \cline{2-7}
&$O_{sin}$ &31 &31 &26 &38 &49 \\ \hline
\end{tabular}

\end{center}
\end{frame}

\end{document}